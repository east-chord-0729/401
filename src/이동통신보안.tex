\chapter{Mobile Telecommunication Security}

\newpage
\section{Cryptography Primitive}

암호화의 기본 기법으로 대칭 키 암호화, 비대칭 키 암호화, 전자 서명 그리고 해시가 있다.
이를 바탕으로 보안 시스템을 구현할 수 있다.

\subsection*{Symmetric Key Encryption}

대칭 키 암호는 암호화 키와 복호화 키가 같은 암호 알고리즘을 뜻한다. 따라서
송신자와 수신자는 서로 같은 키를 공유해야 한다. 대칭 키 암호 통신 과정은
\ref{equ:sym}과 같다.
\begin{equation}
  \begin{split}
    \alice&: \ct = \enc(\key, \pt) \\
    \alice \to \bob&: \id(\alice), \ct \\
    \bob&: \pt = \dec(\key, \ct)
  \end{split}
  \label{equ:sym}
\end{equation}

블록 암호와 스트림 암호가 대칭 키 암호에 속한다. 대표적인 대칭 키 암호에는 AES,
SEED, ARIA, LEA가 있다. 대칭 키 암호는 공개 키 암호보다 빠르기에 대용량 데이터를
암호화하는데 이점이 있으나, 다음과 같은 키 분배 문제를 가지고 있다는 단점이
있다.
\begin{itemize}
  \item 같은 키를 어떻게 공유할 것인가? 공유 도중에 도청자가 키를 탈취할 수
        있다.
  \item 많은 사람들이 통신을 할 경우, 사람마다 보유해야 하는 키가 많아진다.
        ($x$명의 경우, $\binom{x}{2}$)
\end{itemize}

키 분배 문제를 해결하기 위한 시도로 키 분배 센터(KDC)라는 신뢰가능한 제 3자를
이용할 수 있다. 커버러스라고 불리는 통신이 이를 사용한다. KDC를 이용한 통신
과정은 \ref{equ:kdc}와 같다. 그러나 이 또한 근본적으로 키 분배 문제를 해결하지
못한다.
\begin{equation}
  \begin{split}
    \alice \to \carol&: \id(\alice), \id(\bob) \\
    \carol \to \alice&: \ct_1 \samp \enc_{\key_{ac}}(\key_{ab}),
    \ct_2 \samp \enc_{\key_{bc}}(\key_{ab}) \\
    \alice&: \key_{ab} \samp \dec_{\key_{ac}}(\ct_1),
    \ct \samp \enc_{\key_{ab}}(\pt) \\
    \alice \to \bob&: \ct, \ct_2 \\
    \bob&: \key_{ab} \samp \dec_{\key_{bc}}(\ct_2),
    \pt = \dec_{\key_{ab}}(\ct).
  \end{split}
  \label{equ:kdc}
\end{equation}

\subsection*{Asymmetric Key Encryption}

비대칭 키(혹은 공개 키로도 불림)는 대칭 키 암호와 다르게 암호화 키(공개 키)와
복호화 키(비밀 키)가 다른 암호 알고리즘을 뜻한다. 두 키는 서로 수학적으로
연계되어, 하나의 키로 암호화 되었다면, 다른 하나의 키로만 복호화 할 수 있다.
통신 과정은 \ref{equ:asy}와 같다. 비대칭 키 암호의 단점은 대칭 키 암호보다
느리기 때문에 작은 양의 데이터를 처리할 때 사용한다.
\begin{equation}
  \begin{split}
    \bob \to \alice&: \ct = \enc_{\pkey_{\alice}}(\pt) \\
    \alice&: \pt = \dec_{\skey_{\alice}}(\ct)
  \end{split}
  \label{equ:asy}
\end{equation}

공개 키 암호의 장점은 송신 시 자신이 통신하기로 한 송신자인지를 \ref{equ:auth}와 같이 인증할 수 있다.
\begin{equation}
  \begin{split}
    \bob \to \alice&: \ct = \enc_{\skey_{\alice}}(\pt) \\
    \alice&: \pt = \dec_{\pkey_{\alice}}(\ct)
  \end{split}
  \label{equ:auth}
\end{equation}

일반적으로, 대용량의 데이터를 암호화하여 송신하고 싶을 때, 비대칭 키 암호를
이용하여 대칭 키를 암호화하여 공유하고, 대칭 키를 이용하여 데이터를 암호화하는
하이브리드 기법을 이용한다. 통신 과정은 \ref{equ:hybrid}와 같다.
\begin{equation}
  \begin{split}
    a : kab \\
    a : c_1 = e(kab, M) \\
    a : c_2 = e(pk_b, k_ab) \\
    a to b : c_1, c_2 \\
    b : k_ab = d(sk_b, c_2) \\
    b : m = d(k_ab, c_1)
  \end{split}
  \label{equ:hybrid}
\end{equation}

암복호화, 전자서명(DSS), 키교환(DH). RSA, ECC는 전부 가능
RSA는 데탑에서는 사용할만한데...스마트폰이나 iot에서 사용할 때 문제가 생김
오버헤드가 너무 커. ECC는 이걸 해결해. TLS는 예전에는 RSA를 쓰다가 ECC로 바꿈.

\subsection*{Hash Function and Digital Signature}

해시함수는 임의 길이의 데이터를 고정 길이 데이터로 변환하는 함수이다. 일방향
함수이며, 해시값은 통계적으로 유일. 해시함수는 데이터의 무결성(혹은 메시지 인증)을 검증하는데
활용됨.

\begin{equation}
  \begin{split}
    \alice &: \sig = \enc(\skey_{\alice}, \hash(\pt)), \ct = \enc(\pkey_{\bob}, \pt) \\
    \alice \to \bob &: \\
    \bob &: m = d(sk_b, c), check h(m) = d(pk_a, ehash)
  \end{split}
\end{equation}

% 종단간 보안이란?

수동적인 공격자는 통신자 간에 전달되는 데이터를 도청만 하는 공격자를 말한다.
도청을 막기위해 기밀성이 필요하며, 통신자들은 공격자가 도청을 하는지 안하는지 알
수 없다. 반면에 능동적인 공격자는 도청뿐만 아니라 데이터를 직접 건드리는
공격자를 말한다.

mac은 공개키 기반이 아니라는 장점이 있다. 두 대상이 비밀키를 공유하고 있을 때,
메시지를 보낼 때 태그도 함께 보냄. 해시값을 대칭 키로 암호화.
\begin{equation}
  \begin{split}
    \alice &: \mac = \enc(\key_{\alice \bob}, \hash(\pt)) \\
    \alice \to \bob &: \pt, \mac \\
    \bob &: \text{check } \dec(\key_{\alice \bob}, \mac) \issame \hash(\pt)
  \end{split}
\end{equation}

암호 기술 없이 사용하는 mac도 있음. 아래 참고. $s_{\alice \bob}$은 사전에 공유한 비밀 값.
최근에는 hmac을 사용함.
\begin{equation}
  \begin{split}
    \alice &: \mac = \hash(\pt \parallel s_{\alice \bob}) \\
    \alice \to \bob &: \pt, \mac \\
    \bob &: \text{check } \hash(\pt \parallel s_{\alice \bob}) \issame \mac
  \end{split}
\end{equation}

mac는 부인방지(자신이 보낸게 아니라고 시치미 때는 것을 막는 것)가 안됨. 왜냐하면
MAC값은 $\alice$뿐만 아니라 $\bob$도 만들 수 있기 때문. 이를 해결할 수 있는
방법으로 전사서명이 있음. 전자 서명은 인증에 추가로 부인방지까지 가능함.
전자서명 과정은 다음과 같음.
\begin{equation}
  \begin{split}
    \alice &: \text{generate } \key_{\alice \bob} \\
    \alice &: \sig = \enc(\skey_{\alice}, \hash(\pt)), \ct_{\pt} = \enc(\key_{\alice \bob}, \pt),
    \ct_{\key} = \enc(\pkey_{\bob}, \key_{\alice \bob}) \\
    \alice \to \bob &: \ct_{\pt}, \ct_{\key}, \sig \\
    \bob &: \key_{\alice \bob} = \dec(\skey_{\bob}, \ct_{\key}) \\
    \bob &: \pt = \dec(\key_{\alice \bob}, \ct_{\pt}) \\
    \bob &: \text{check } \hash(\pt) \issame \dec(\pkey_{\alice}, \sig)
  \end{split}
\end{equation}

중간자 공격, 전달하는 메시지를 위조. 공개키를 자신의 공개키로 위조. 서명도
자신의 개인키로 서명. 그러면 받는 사람은 서명 검증 단계에서 valid를 얻음. 문제:
이 공개키가 진짜로 내가 통신하고자 하는 사람의 것인가? 해결시도: CA라고 하는
신뢰하는 제 3자에게 공개 키를 요구하는 방법. -> 근본적인 해결 X 이것도 통신중에
바뀔 수 있음.

해결방법: 이 공개키는 내거라는 인증서를 만든다. 신뢰할 수 있는 제 3자가 인증서
발행. 인증서에는 크게 소유자의 ID, 공개 키, 그리고 발급자의 서명(위조 방지)가
있다.

안전한 공개키 암호 기술을 사용하기 위해 공개키 기반 구조(PKI, 인증서를 사용하는
인프라)가 필요.

CA(certification authority): 인증기관. -> 인증서 관리.
RA(registration authority): 등록기관. -> 사용자 신분 확인.

Y가 발행한 X의 인증서 Y<<X>>.

root 인증서 : 발급자와 요청자가 같은 사람.

root 인증서를 가지고 CA의 기능을 수행한다?
크롬에 설치?

CA가 하나만 있는건 현실적으로 어려움.
한국 사람이 중국의 CA를 신뢰해서 알리 익스프레스에서 거래가 가능한가? 여러 CA가 신뢰관계를 쌓아야함.
-> X1이 X2의 인증서를 발급하여 A에게 전달. (신뢰관계 형성)
-> A는 X2의 공개키를 신뢰할 수 있음.

인증서를 취소해야 할 때. 유효기간의 만료. 인증서가 사라져서 재발급 해야 할 때. 등

인증서 취소 목록(CRL)
-> 취소 목록 생성 일시, 다음 생성 일시 (스케줄), 취소된 인증서 목록(일련번호, 취소 일시 등), CA의 서명.

시간이 지날수록 CRL 크기 증가.
시간차공격 가능.
단일 CRL일 경우 관리가 어려움.

인증서 검증 방법: 1. 유효한가? 2. 경로 검증 3. 인증서 용도 등...

자기가 자기한테 인증서를 발급한다. -> 루트인증서. 내 공개키를 신뢰하게 만드는 상위 CA가 없을 때.

미션?

Diffie-Hellman(이하 DH) 키 교환은 암호 키를 교환하는 하나의
방법으로, 두 사람이 암호화되지 않은 통신망을 통해 공통의 비밀 키 $\key$를 공유할 수
있도록 한다. 휫필드 디피와 마틴 헬만이 1976년에 발표하였다. 앨리스 $\alice$와 밥
$\bob$이 공개된 통신망에서 DH 키 교환을 하기 위해서는 그림
\ref{fig:diffie}와 같은 절차를 거친다.

\begin{figure}[ht]
  \centering
  \begin{tikzpicture}

    % alice
    \node at (0, 0) (alice) {$\alice$};
    \node at (0, -1) (choose_pg) {choose $p, g$};
    \node at (0, -2) (choose_a) {choose $a$};
    \node at (0, -3) (compute_ya) {$A := g^a \mod p$};
    \node at (0, -5) (get_yb) {$B$};
    \node at (0, -6) (compute_yab) {$\key := B^a \mod p$};

    % bob
    \node at (7, 0) (bob) {$\bob$};
    \node at (7, -1) (get_pg) {$p, g$};
    \node at (7, -3) (get_ya) {$A$};
    \node at (7, -4) (choose_b) {choose $b$};
    \node at (7, -5) (compute_yb) {$B := g^b \mod p$};
    \node at (7, -6) (compute_yba) {$\key := A^b \mod p$};

    % line
    \draw[->] (choose_pg) -- (get_pg);
    \draw[->] (compute_ya) -- (get_ya);
    \draw[->] (compute_yb) -- (get_yb);

  \end{tikzpicture}
  \caption{Diffie-Hellman Key Exchange}
  \label{fig:diffie}
\end{figure}

마지막 단계에서 $\alice$와 $\bob$은 $g^{ab} \mod p$라는 공통의 $\key$를
공유하게 된다. $\alice$와 $\bob$ 외에는 $a, b$를 알 수 없으며, $p, g, A, B$만 알
수 있다. DH 키 교환은 통신을 하는 대상과 비밀 정보를 공유할 수 있지만,
상대방에 대한 인증은 보장되지 않으며 중간자 공격이 가능하다. $\alice$와 $\bob$이
상대방에 대한 인증을 하지 못할 경우, 공격자 $\eav$는 그림 \ref{fig:diffie_mitm}과 같이
중간에서 통신을 가로채 $\alice$와 $\eav$, 그리고 $\eav$와 $\bob$ 사이에 각각 두 개의
DH 키 교환을 생성하고, $\alice$와 $\bob$이 각각 서로와 통신을 하는 것처럼
위장할 수 있다.

\begin{figure}[ht]
  \centering
  \begin{tikzpicture}

    % alice
    \node at (0, 0) (alice) {$\alice$};
    \node at (0, -1) (choose_pg) {choose $p, g$};
    \node at (0, -2) (choose_a) {choose $a$};
    \node at (0, -3) (compute_ya) {$A := g^a \mod p$};
    \node at (0, -9) (get_yb) {$B$};
    \node at (0, -10) (compute_yab) {$\key_2 := B^a \mod p$};

    % eav
    \node at (5, 0) (eav) {$\eav$};
    \node at (5, -1) (eav_pg) {$p, g$};
    \node at (5, -3) (eav_ya) {$A$};
    \node at (5, -4) (choose_e1) {choose $e_1$};
    \node at (5, -5) (compute_ye1) {$A := g^{e_1} \mod p$};
    \node at (5, -7) (eav_yb) {$B$};
    \node at (5, -8) (choose_e2) {choose $e_2$};
    \node at (5, -9) (compute_ye2) {$B := g^{e_2} \mod p$};

    % bob
    \node at (10, 0) (bob) {$\bob$};
    \node at (10, -1) (get_pg) {$p, g$};
    \node at (10, -5) (get_ya) {$A$};
    \node at (10, -6) (choose_b) {choose $b$};
    \node at (10, -7) (compute_yb) {$B := g^b \mod p$};
    \node at (10, -10) (compute_yba) {$\key_1 := A^b \mod p$};

    % line
    \draw[->] (choose_pg) -- (eav_pg);
    \draw[->] (eav_pg) -- (get_pg);
    \draw[->] (compute_ya) -- (eav_ya);
    \draw[->] (compute_ye1) -- (get_ya);
    \draw[->] (compute_yb) -- (eav_yb);
    \draw[->] (compute_ye2) -- (get_yb);

  \end{tikzpicture}
  \caption{Man in the Middle Attack in Diffie-Hellman Key Exchange}
  \label{fig:diffie_mitm}
\end{figure}

따라서, 공개 키 $A$ 또는 $B$를 건네 받을 경우, 이 공개 키가 $\alice$ 또는 $\bob$의 공개 키가 맞는지
인증서를 이용하여 확인해야 한다.

공동인증기관(루트 CA) -- 상호인증 -- 정부 인증 체계

%******************************************************************************%
\newpage
\section{Kerberos Protocol}

인증을 하는 서버가 3개 나옴. 그래서 커버러스라고 함.

대칭키 암호를 사용한 사용자 인증.

상호인증: 통신 참여자가 각자 상대방 신분을 확인함. 키 교환 가능.

기밀성을 보장하는가? 메시지 재전송 공격을 막을 수 있는가? (적시성.)

옛날에 보낸 메시지를 도청해서 보관하다가 나중에 보내서 송신자인 척 함.

해결 방법:
1. 서로가 메시지를 보내고 받는 횟수를 카운팅함.
2. 시간정보를 넣음.
3. 랜덤값을 송신자한테 보내고 이걸 암호화하라고 함.

\subsection{Needham Schroeder Protocol}

니덤-슈로더 프로토콜. KDC 기반 세션키 분배 프로토콜. 앨리스와 밥은 KDC를 신뢰함.
앨리스와 밥은 각자 Ka Kb를 가지고 있음. KDC는 둘 다 가지고 있음.
\begin{equation}
  \begin{split}
    \alice \to \kdc &: \id(\alice) \parallel \id(\bob) \parallel N_1 \\
    \kdc \to \alice &: \enc(\key_{\alice},
    \key \parallel \id(\bob) \parallel N_1 \parallel
    \enc(\key_{\bob}, \key \parallel \id(\alice))) \\
    \alice \to \bob &: \enc(\key_{\bob}, \key \parallel \id(\alice))
  \end{split}
\end{equation}

\newpage
\subsection{Kerberos Protocol}

커버러스는 제 3의 신뢰된 인증서버를 통해 개방된 분산 환경에 존재하는 사용자간의
인증과 세션키 교환을 지원하는 인증 서비스이다.

임의의 클라이언트 $\client$가 서버 $\server$에 접속하기 위해서는,
자격이 없는 클라이언트 $\client'$가 신분 위장 후 $\server$에 접속하는 시도를 막기위해
인증을 시도해야한다.

기호를 다음과 같이 정리한다.
\begin{itemize}
  \item $\client$: 클라이언트(사용자)
  \item $\server$: 어플리케이션 서버
  \item $a$: 인증 서버
  \item $\id_x$: $x$에 대한 식별자
  \item $\pw_x$: $x$가 사용하는 패스워드
  \item $\netadr_x$: $x$의 네트워크 주소
  \item $\key_{x, y}$: $x$와 $y$가 공유하는 비밀키
\end{itemize}

초기 커버러스
1. 패스워드가 평문
2. 재전송공격
3. 다른 서버에 인증하기 위해서는 티켓을 다시 받아야함.
4. 일방향 인증

TGS를 이용한 커버러스
1. 티켓의 유효기간에 따른 트레이드오프
2. 일방향 인증

\subsection{Kerberos Protocol V4}

롱텀 키: 계속 쓰는 키.
세션 키: 잠깐 쓰고 버리는 키.

C는 AS를 신뢰할 수 있음. Kc는 C하고 AS만 공유하는 데, 평문에 IDtgs 로
인증하면서, TS2로 재전송 공격도 막으므로 신뢰 가능함.

\begin{figure}[H]
  \centering
  \begin{tikzpicture}
    \matrix[column sep=50mm,row sep=5mm]
    {
      \node {$\cl$};   & \node {$\au$};  \\
      \node (L1) {}; & \node (R1) {}; \\
       & \node (R2) {$\tk := \enc_{\key_{\au\tgs}}$};
                       & \node {$\tgs$};  \\
                       & \node {$\sv$};  \\
    };
    \graph
    {
      (L1) ->["$\id_{\cl}, \id_{\tgs}, \ts$"] (R1);
    };
  \end{tikzpicture}
  \caption{Kerberos version 4 protocol}
  \label{fig:ker_v4}
\end{figure}

\subsection{Kerberos Protocol V5}

V4의 한계:
DES 사용. 그러나 DES는 수출제한이 걸려있음.
주소는 IP 주소만 사용.
메시지 바이트 순서 고정.
등등등

기술적 결점:
이중 암호화: 암호화로 티켓을 만들고, 그 티켓을 또 암호화함. 굳이 두번씀.
PCBC 운용 모드(비표준) 사용.
세션키 중복 사용.
패스워드 추측 가능.

\begin{figure}[H]
  \centering
  \begin{tikzpicture}
    \matrix[column sep=50mm,inner sep=2mm]
    {
      \node {$\cl$};   & \node {$\au$};  \\
      \node (send) {}; & \node (get) {}; \\
                       & \node {$\tgs$};  \\
                       & \node {$\sv$};  \\
    };
    \graph
    {
      (send) ->["$a, b, c, d, e$"] (get);
    };
  \end{tikzpicture}
  \caption{}
  \label{fig:mac}
\end{figure}

%******************************************************************************%
\newpage
\section{Secure Socket Layer, SSL}

이 절에서는 보안 소켓 레이어(SSL)에 대해 설명한다.

\subsection*{Background}

소켓은 네트워크 통신을 위해 사용하는 라이브러리. SSL이란 소켓을 사용하는 임의의
응용프로그램들을 위해 설계되었다. SSL의 중요한 목적은 두 통신 객체 사이에
기밀성과 무결성을 지원하는 것이다. SSL은 후에 TLS로 발전하였다. SSL은 통신객체
인증, 데이터 기밀성, 데이터 인증 및 무결성, 압축, 세션키 생성 및 분배, 보안속성
협의(알고리즘, 키길이 등) 서비스를 제공한다.

SSL은 TCP/IP 4계층에서 전송 계층과 응용 계층 사이에서 동작한다.

\subsection*{Record Protocol}

\def\record{\textsf{Record}}

SSL 프로토콜은 우선 레코드 프로토콜이 동작한다. 레코드 프로토콜은 응용 데이터의
기밀성과 무결성을 제공한다. 과정은 다음과 같다.
\begin{quote}
  \centering
  fragment $\to$ compress $\to$ add MAC $\to$ encrypt $\to$ append SSL record
  header.
\end{quote}

\subsection*{Hand Shake Protocol}

핸드셰이크 프로토콜은 SSL 레코드 프로토콜에서 사용될 암호화 알고리즘과 압축
알고리즘을 정하고, 객체(서버와 클라이언트)를 인증하며, 암호화 키를 생성하기 위한
master secret 생성한다.
\begin{equation}
  \begin{split}
    \cl \to \sv &: \textsf{client hello} \\
    \sv \to \cl &: \textsf{server hello} \\
    \cl \to \sv &: \textsf{premaster secret, change cipher spec, finished} \\
    \sv \to \cl &: \textsf{change cipher spec, finished}
  \end{split}
\end{equation}
\begin{itemize}
  \item \textsf{client hello}: 클라이언트는 지원하는 최고 TLS 프로토콜 버전,
  난수, 희망하는 세션 아이디, 제안된 암호 모음 목록 및 제안된 압축 방법을 지정하여 보냅니다.
  \item \textsf{server hello}: 서버는 클라이언트가 선택한 프로토콜 버전, 난수, 세션 아이디,
  암호 제품군 및 압축 방법을 포함하여 보냅니다.
  \item \textsf{certificate}: 서버는 클라이언트가 서버의 공개키를 신뢰할 수 있도록
  인증서를 전송합니다.(중간자 공격 방지).
  \item \textsf{server hello done}: 서버는 핸드셰이크 협상이 완료되었음을 나타냅니다.
  \item \textsf{client key exchage}: 클라이언트는 \textsf{premaster secret}을
  서버 인증서의 공개 키를 사용하여 암호화하여 전송합니다.
  
  change cipher spec: 협상 성공 여부(선택한
  알고리즘 사용하겠다!), finished(mac값이라고 생각하셈.) = prf(matser secret,
  client finish, H(handsake messages)). request를 받았다면, certificate와
  certificate verify를 보냄. 개인키로 암호화해서 보냄. (전자서명.) 난수도
  따끈따끈한 난수라 재전송 공격 방지. master secret = seed1 seed2 premaster
  \item \textsf{change cipher spec}:
  \item \textsf{finished}: finish 받음. 개인키로 복호화 해서 premaster secret 얻음. -> master secret 생성.
  spec 확인해서 우리가 사전에 협약한 알고리즘을 쓰는지 확인. -> prf 돌려서 finish 생성.
  이제 내가 키를 증명해야 하니까 새로운 finish 생성. (두개는 서로 다른 Finish임. 입력이 좀 다름.)
  spec도 보냄. client: spec 확인. 알고리즘 쓴다는걸 확인해. 그럼 finish 생성해서 같은지 확인.
\end{itemize}

sertificate request: 클라이언트 인증을 위해 요구함. 기존은 서버만 인증하는 구성임.

마스터 시크릿이 만들어졌어. 이제 데이터를 암호화 해야 함. 레코드 프로토콜에서 맥값, 암호화 등등 해야 하니까.
그걸 위한 키들은 키블럭에 있음. 이 키블럭은 마스터 시크릿으로 만드는 거임.

클라이언트/서버 맥 키,  클라이언트/서버 키, 클라이언트/서버 초기화 벡터.

1.3
최신암호. 1라운드 트립. 롱텀키를 뺐겨도 세션키는 안전함(순방향비밀성). 신분노출방지.

ECDH를 사용하면 순방향비밀성을 지킬 수 있다. server key exchange 때문. 이 때 전자서명도 한번에 보내므로
finish가 아니라 1라운드에서 인증이 된다.

%******************************************************************************%
\newpage
\section{Transport Layer Security, TLS}

레코드 프로토콜: 분할 -> 암호화 -> 인증(AEAD).
기존에는: 분할 -> 압축 -> 인증 -> 암호화.

핸드셰이크: 두 객체가 인증하기 위한 프로토콜. 키교환, 알고리즘 협상.
성능향상: 2라운드 트립 -> 1라운드 트립.
보안: 알고리즘 향상. 

키교환 옵션: 
  풀(서로가 처음 통신할 때, 완전순방향), 
  프리셰어드(이전 세션에서 공유된 키를 통해 키교환)


키쉐어: DH 키가 있음. -> tkchs를 생성: 핸드셰이크를 암호화할 키.
   
중간자 공격 가능: 첫번째 키교환은 가능 -> 늦게 알게됨(늦게라도 검증은 됨), 근데 뭐 공격자가 어마무시한 걸 알 가능성이 낮다고 판단.
, 두번째 공개키는 인증서 때문에 불가능.
자원 고갈 공격: 계속 키 생성 요청.

