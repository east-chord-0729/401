\chapter{Mathematical Analysis}

% 단답형 질문은 쪽지, 복잡한 질문은 e-mail. (평일 9:00 ~ 17:00), 학번 이름 학과 분반 밝힐 것.
% e-mail: kagness@kookmin.ac.kr
% 40 40 과제 10 출석(호명) 10
% 쉬는날 비대면.
% 과제는 자필로 + 종이로. 중간 5번 기말 5번 있음. 풀이과정, 노력점수 있음.
% 지각 -1점
% 결석: 주임교수, 대학장 직인필.

\section{Axiom}

\begin{definition}

  \label{def:field}
\end{definition}

\begin{axiom}
  $\RR$ is a field.
  \label{axm:field}
\end{axiom}

\begin{axiom}
  Let $P$ be a nonempty subset of $\RR$ which we define as the set of positive
  real numbers. Then $P$ have following axioms:
  \begin{itemize}
    \item Axiom 1. If $a, b \in P$, then $a + b \in P$.
    \item Axiom 2. If $a, b \in P$, then $a \cdot b \in P$.
    \item Axiom 3. If $a \in \RR$ then only one of the following holds: $a \in P$
          or $-a \in P$ or $a = 0$.
  \end{itemize}
  \label{axm:order}
\end{axiom}

\begin{property}
  Let $P$ be a nonempty subset of $\RR$ which we define as the set of positive
  real numbers. Then $P$ have following properties:
  \begin{itemize}
    \item $\forall a, b, c \in \RR$, exactly one of $a < b$, $a = b$ and $a > b$
          is true. 순서가 있으니 $<, >, =, \ge, \le$ 등을 사용할 수 있음.
    \item $\forall a, b, c \in \RR$, if $a < b$ and $b < c$ then $a < c$.
    \item $\forall a, b \in \RR$, if $a \le b$ and $a \ge b$ then $a = b$.
    \item $\forall a, b, c \in \RR$, if $a < b$ then $a + c < b + c$.
    \item $\forall a, b, c \in \RR$, if $a < b$ and $c > 0$ then $ac < bc$.
  \end{itemize}
\end{property}

\section{Bound}

\begin{definition}
  Let $X \subseteq \RR$ and $ X \neq \varnothing$. $X$ is \emph{bounded above}
  if $\forall x \in X, \exists a \in \RR:a \geq x$. $a$ is \emph{upper bound} of $X$.
  $X$ is \emph{bounded below} if $\forall x \in X, \exists b \in \RR:b \leq x$.
  $b$ is \emph{lower bound} of $X$.
  \label{def:bound}
\end{definition}

\begin{definition}
  Let $a$ is upper bound of $X$. $a$ is a \emph{supremum} if for all upper bound
  $a'$ of $X$, $a' \leq a$. $a$ is denoted $\sup X$.
  Let $b$ is lower bound of $X$. $b$ is a \emph{infimum} if for all lower bound
  $b'$ of $X$, $b' \geq b$. $a$ is denoted $\inf X$.
  \label{def:sup}
\end{definition}

\begin{theorem}(completeness axiom)
  Let $X$ is bounded above. $\sup X$ is uniquely exist.
  Let $X$ is bounded below. $\inf X$ is uniquely exist.
  \label{pro:sup_uniq}
\end{theorem}
\begin{proof}

\end{proof}

\begin{theorem}
  Let $a$ is upper bound of $X$
  $X$가 위로 유계이고, $a$가 $X$의 상계일 때, $a = \sup X$라면, 모든 $\eps < 0$에 대해
  $a - \eps < x \leq a$를 만족하는 $x \in X$가 존재하며, 그 역도 성립한다.
  \label{pro:bound1}
\end{theorem}
\begin{proof}
  $(\rif)$ 여기서는 대우 명제를 증명한다. 모든 $\eps < 0$에 대해 $a -
    \eps < x \leq a$를 만족하는 $x$가 존재하지 않는다고 가정하자. 이는 즉
  모든 $x$에 대해 $a - \eps < x$를 만족하므로, $a - \eps$은 $X$의
  상계가 된다. $a - \eps < a$이기 때문에 $a$는 $\sup X$가 아니다. \\
  $(\lif)$ 모든 $\eps < 0$에 대해 $a - \eps < x \leq a$인 $x$가
  존재하므로 $a - \eps$은 상계가 아니다. 이는 $a$보다 작은 모든 실수가
  상계가 아니라는 뜻이므로, $a$는 상한이다.
\end{proof}

$\sup X = \infty$라면, $X$는 위로 유계가 아니다.

$X = \emptyset$이면, $\sup = -\infty$.
$\forall r \in \RR$, $x \in X \implies x \leq r$이다.(p이면 q이다 에서 p가
거짓이면, 이 조건문은 무조건 참).

\begin{definition}
  $X$가 위로 유계가 아니면, $\sup X = \infty$이라 한다.
  $X$가 아래로 유계가 아니면 $\inf X = -\infty$이라 한다.
  \label{def:not_bound}
\end{definition}

실수 $\RR$은 위로 유계도, 아래로 유계도 아니다.

아르키메데스의 성질
\begin{theorem}
  모든 $a, b \in \RR, a > 0$에 대하여, $na > b$이 성립 되는 적당한 $n \in \NN$가 존재한다.
  \label{thm:arc}
\end{theorem}

\begin{proof}
  (역이 모순이라는 것을 증명)
  모든 자연수 $n$가 $na \le b$를 만족하는 $a > 0$이 존재한다고 가정하자.
  $b$는 $A = \set{na : n \in \NN}$의 상계이고, $A$는 위로 유계이다.
  완비성의 공리에 의해, $\sup A$는 존재한다.
  즉 $(n + 1) \cdot a \le \sup A$를 만족한다. 이는
  $na < \sup A - a$이므로,  $\sup A - a$는 $A$의 상계이다.
  상한의 정의에 의해, $a \le 0$이므로,

  정리 \ref{pro:bound1}에 의해, 모든 $\eps > 0$에 대해
  $\sup A - \eps < n \le \sup A$를 만족하는 $n$이 존재한다.
\end{proof}

이를 통해 알 수 있는 것. $0 < a < b$에 대해 $na > b$이 성립하는 $n$이 존재한다.
$a$가 아무리 작고 $b$가 아무리 커도, $a$를 계속 더하면 $b$를 넘을 수 있음.

자연수 집합 $\NN$은 실수 $\RR$에서 위로 유계가 아니다.


\subsection*{Exercises}

\begin{exercise}
  모든 $n \ge 4$에 대해, $n! > 2^n$임을 보여라.
\end{exercise}

\begin{solution}
  수학적 귀납법으로 보인다.
  $4! = 24 > 2^4$이므로 $n = 4$일 때 성립.
  $k > 4$에 대하여, $k! > 2^k$가 성립한다고 가정하자. 그러면
  $(k + 1)! > 2^k \cdot (k + 1) > 2^k \cdot 2 > 2^(k + 1)$이므로 증명은 완성된다.
\end{solution}

\begin{exercise}

\end{exercise}

\begin{solution}
  $A_2 = A_2^o and A_2^e$이다.
  $\sup A_2^o = \frac{1}{2}$
  $\inf A_2^e = -\frac{1}{2}$ 뭐 대충 이느낌.
\end{solution}

\begin{exercise}
  $A = (-\infty, 3) and (4, 5]$에서의 상한과 하한을 구해라.
\end{exercise}

\begin{solution}
  $\sup A = 5, \inf A  = -\infty$
\end{solution}

\begin{exercise}
  $a, b \in \RR$이라 하자. 임의의 $\eps$에 대하여, $\abs{a-b} < \eps$이면, $a = b$임을 보여라.
\end{exercise}

\begin{solution}
  $A = \set{\eps \in \RR : \eps > 0}$이라 하자. $A = (0, \infty)$이다.
  $\abs{a - b}$는 $A$의 하계이다. $A$는 아래로 유계이다.
  완비성의 공리에 의해, $\inf A$가 유일하게 존재한다. $\inf A \ge \abs{a - b}$이다.
  $\inf A = 0$이다. $\abs{a - b} \ge 0$이므로, $\abs{a - b} = 0$이다.
\end{solution}

\begin{exercise}
  (베르누이 부등식)$x > -1$이면 모든 자연수 $n$에 대하여 $(1 + x)^n \ge 1 + nx$임을 보여라.
\end{exercise}

\begin{solution}
  PMI(?) $(1 + x)^1 = 1 + x \ge 1 + 1x$이므로 성립.
  $(1 + x)^k \ge 1 + kx$가 성립한다고 가정하자.
  $(1 + x)^(k+1) = (1 + x)^k \cdot (1 + x) \ge (1 + kx) \cdot (1 + x)
    = 1 + kx + x + kx^2 \ge 1 + kx + x = 1 + (k + 1)x$가 성립하므로 증명은 완성된다.
\end{solution}

%******************************************************************************%
\section{Sequence}

\begin{definition}
  Let $a_n$ is a sequence. $a_n$ \emph{converges} if
  \begin{equation}
    \forall \eps \in \RR_{> 0}, \exists N \in \NN: \forall n \in \NN_{\ge N}, \abs{a_n - l} < \eps.
  \end{equation}
  $l \in \RR$ is a \emph{limit} of the sequence $a_n$, which is written $\limn
    a_n = l$.
  \label{def:lim}
\end{definition}

% 예를 들어, $a_n = \frac{1}{n}$이라고 할 때, 아르키메데스의 성질에 의해 $a_n \to 0$이다.
% $a_n = 2 + \frac{1}{2^n}$이라고 할 때, 베르누이 부등식을 사용하여, $a_n \to 2$임을 알 수 있다.

\begin{theorem}
  If the limit of a sequence exists then it is unique.
  \label{thm:lim_uniq}
\end{theorem}

\begin{proof}
  Let $l_1, l_2$ are limit of the sequence $a_n$. Then
  \begin{equation}
    \begin{split}
      &\forall \eps > 0, \exists N_1:
      \forall n \ge N_1, \abs{a_n - l_1} < \frac{\eps}{2} \\
      &\forall \eps > 0, \exists N_2:
      \forall n \ge N_2, \abs{a_n - l_2} < \frac{\eps}{2}.
    \end{split}
  \end{equation}
  Let $N = \max(N_1, N_2)$. Then
  \begin{equation}
    \forall n \ge N, \abs{l_1 - l_2}
    = \abs{l_1 - a_n + a_n - l_2}
    \le \abs{a_n - l_1} + \abs{a_n - l_2}
    < \frac{\eps}{2} + \frac{\eps}{2} = \eps
  \end{equation}
  It follows that $\abs{l_1 - l_2} = 0$ and $l_1 = l_2$. It completes the proof.
\end{proof}

% \footnote{역은 성립하지 않는다. $\set{(-1)^n}$가 그 예시이다.}

\begin{theorem}
  If the limit of a sequence exists then the sequence is bounded.
  \label{thm:lim_bou} % limit bound
\end{theorem}

\begin{proof}
  Let $l$ is limit of the sequence $a_n$. Then
  \begin{equation}
    \forall \eps > 0, \exists N: \forall n \ge N, \abs{a_n - l} < \eps.
  \end{equation}
  It follows that
  \begin{equation}
    \abs{a_n} = \abs{a_n - l + l} \le \abs{a_n - l} + \abs{l} < \eps + \abs{l}.
  \end{equation}
  Let $M = \max(a_1, \cdots, a_{N-1})$. Then
  \begin{equation}
    \forall n, a_n \le \max (M, \eps + \abs{l}).
  \end{equation}
  It completes the proof.
\end{proof}

% \begin{proposition}
%   $a_n$이 유계수열이고 $y_n$이 0으로 수렴하면, 수열 $a_n y_n$은 0으로 수렴한다.
% \end{proposition}

% \begin{proof}
%   모든 $n \in NN$에 대해 $\abs{a_n} \le M$인 $M$이 존재한다. 모든 $\eps > 0, n
%     \in \NN$에 대해 $n \ge N$을 만족하는 $N$이 존재하면, $\abs{y_n - 0} <
%     \frac{\eps}{M}$이다.

%   다음으로 증명은 완성된다.
%   $\abs{a_ny_n - 0} = \abs{a_ny_n} < M \cdot \frac{\eps}{M} < \eps$.
% \end{proof}

% 과제 1번.

% \begin{proof}
%   for all $\eps > 0$, and for all $n \in \NN$, exist $N$ s.t. $n \ge N$ and
%   $\abs{\frac{1}{n^2 + 1} - 0} < \eps$.

%   $n^2 > n^2 > N^2 \implies \frac{1}{n^2 + 1} < \frac{1}{N^2} = \eps$
% \end{proof}

\begin{theorem}[Squeeze theorem]
  \label{thm:limit_squeeze}
  Let $a_n \to l, y_n \to l$. Then
  \begin{equation}
    \forall n, a_n \le z_n \le y_n \implies z_n \to l.
  \end{equation}
\end{theorem}

\begin{proof}
  Since $a_n \to l, y_n \to l$, we have
  \begin{equation}
    \begin{split}
      \forall \eps > 0, \exists N_x: \forall n \ge N_x, -\eps < a_n - l < \eps \\
      \forall \eps > 0, \exists N_y: \forall n \ge N_y, -\eps < y_n - l < \eps.
    \end{split}
  \end{equation}

  Let $N = \max(N_x, N_y)$. since $a_n \le z_n \le y_n$, we completes the proof
  by following
  \begin{equation}
    \forall \eps > 0, n \ge N, -\eps < a_n - l \le z_n - l \le y_n - l < \eps.
  \end{equation}
\end{proof}

% $z_n = \sin / n \to 0$.

% 증가 수열, 감소 수열, 순증가 수열, 순감소 수열, 단조수열.

\subsection{Monotone Sequences}

\begin{definition}
  \label{def:mono}
  the squence $\set{a_n}$ is \emph{monotone increasing sequence} if $\forall n,
    a_n \le x_{n+1}$.
\end{definition}

\begin{lemma}
  If a sequence is increasing and bounded above, then its supremum is the limit.
  \label{lem:mono_limit_1}
\end{lemma}

\begin{proof}
  Let $x := \sup{\set{a_n}}$.
\end{proof}

\begin{lemma}
  If a sequence is decreasing and bounded below, then its infimum is the limit.
  \label{lem:mono_limit_2}
\end{lemma}

\begin{proof}

\end{proof}

\begin{theorem}[Monotone convergence theorem, MCT]
  If a sequence is bounded and monotone, then it converges.
  \label{thm:mono_limit}
\end{theorem}

\begin{proof}

\end{proof}

%***%
\subsection{Subsequences}

\begin{definition}
  $\set{x_{n_k}}$ is a \emph{subsequence} of $\set{a_n}$ if $\set{n_k}$
  is increasing sequence.
  \label{def:sub}
\end{definition}

\begin{theorem}
  If a sequence $\set{a_n}$ has a limit $l$ then all subsequence
  $\set{x_{n_k}}$ of $\set{a_n}$ also have a limit $l$.
  \label{thm:sub_limit}
\end{theorem}

\begin{proof}
  It is clear that $n_1 \ge 1$. Suppose that $n_k
    \ge k$. Since $n_k$ is increasing sequence, we have $n_{k+1} > n_k \ge k$. It
  follows that $n_{k+1} \ge k + 1$ and we have
  \begin{equation}
    n_k \ge k.
  \end{equation}

  Let $a_n \to l$. We completes the proof by following.
  \begin{equation}
    \forall \eps > 0, \exists k: \forall n_k \ge k, \abs{x_{n_k} - l} < \eps.
  \end{equation}
\end{proof}


\subsection{Nested Sequences}

\begin{definition}
  Let $I_n = [a_n, b_n]$, where $\abs{I_n} = b_n - a_n$ denotes the
  \emph{length} of such an interval. One can call $I_n$ a sequence of nested
  intervals, if
  \begin{equation}
    \begin{split}
      \forall n, I_{n + 1} < I_n,
    \end{split}
  \end{equation}
  \label{def:nest}
\end{definition}

\begin{theorem} [Nested intervals theorem]
  Let $I_n = [a_n, b_n]$ is nested. If $(b_n - a_n) \to 0$, then $\exists! x \in
    \RR$ such that $x \in I_n$.
  \label{thm:nest_int}
\end{theorem}

\begin{proof}
  First we prove existence. Note that $a_n$ is bounded. By MCT, $a_n \to \sup
    A$. Let $x = \sup A$. Since $b_n$ is upper bound, $a_n \le x \le b_n$. then $x
    \in I_n$. Now we prove uniqueness. Assume that $y \in I_n$. then $a_n \le y
    \le b_n$. It follows that $0 \le y - a_n \le b_n - a_n$. By squeeze theorem,
  $y - a_n \to 0$. It follows that $a_n \to y = x$.
\end{proof}

\begin{theorem} [Bolzano-Weierstrass theorem]
  bounded sequence has a convergent subsequence.
  \label{thm:nest_bol}
\end{theorem}

\begin{proof}
  Let $\set{a_n}$ is a bounded sequence. Then we have
  \begin{equation}
    \forall n, \exists M > 0: -M \le a_n \le M.
  \end{equation}

  Let $I_1 = [0, M], I_n = [\frac{(2^n - 1)M}{2^n}, M]$. Since $\forall k, I_k
    \supset I_{k + 1}$, $I_n$ is nested. By nested intervals theorem,
  $\abs{x_{n_k} - x} < \frac{M}{2^{n-1}}$.
\end{proof}


\subsection{Cauchy Sequences}

\begin{definition}
  a sequence $\set{a_n}$ is \emph{Cauchy sequence} if
  \begin{equation}
    \forall \eps > 0, \exists N: \forall m > n \ge N, \abs{x_m - a_n} < \eps.
  \end{equation}
  \label{def:cauchy}
\end{definition}

\begin{theorem}
  if a sequence converges, then it is cauchy sequence.
  \label{thm:cauchy_conv}
\end{theorem}

\begin{proof}
  Let $a_n \to l$. Then
  \begin{equation}
    \begin{split}
      \forall \eps > 0, \exists N: \forall n \ge N,
      \abs{a_n - l} < \frac{\eps}{2}, \\
      \forall \eps > 0, \exists N: \forall m \ge N,
      \abs{a_n - l} < \frac{\eps}{2}.
    \end{split}
  \end{equation}

  Let $m > n$. Then
  \begin{equation}
    \abs{x_m - a_n}
    =   \abs{x_m - l + l - a_n}
    \ge \abs{x_m - l} + \abs{a_n - l}
    <   \frac{\eps}{2} + \frac{\eps}{2}.
  \end{equation}
\end{proof}

\begin{theorem}
  A cauthy sequence is bounded.
  \label{thm:cauchy_bound}
\end{theorem}

\begin{proof}
  Let $a_n$ is a cauchy sequence. Then
  \begin{equation}
    \exists N: \forall m > n \ge N, \abs{x_m - a_n} < 1.
  \end{equation}

  Let $n < N$. Then
  \begin{equation}
    \abs{a_n} < \max (\abs{x_1}, \cdots, \abs{x_{N-1}}).
  \end{equation}

  Let $n \ge N$. Then
  \begin{equation}
    \abs{a_n}
    =   \abs{a_n - a_n + a_n}
    \le \abs{a_n - a_n} + \abs{a_n}
    <   1 + \abs{a_n}.
  \end{equation}

  Let $M = \max (\abs{x_1}, \cdots, \abs{x_{N-1}, 1 - \abs{a_n}})$. Then
  $\exists M: \abs{a_n} \ge M$. This completes the proof.
\end{proof}

\begin{theorem} [Cauchy's convergence test]
  A sequence converges if and only if it is cauchy sequence.
  \label{thm:cauchy_test}
\end{theorem}

\begin{proof}
  % (->) 했음.
  % (<-) 코시수열 은 유계수열. 유계수열이면 수렴하는 부분수열을 가짐(PWT). 

  We claim that $x_{n_k} \to l \implies a_n \to l$.

  Let $\set{a_n}$ is a cauchy sequence and $\set{x_{n_k}}$ is a subsequence of
  $\set{a_n}$. Then
  \begin{equation}
    \begin{split}
      &\forall \eps > 0, \exists N_1: \forall m > n \ge N_1, \abs{x_m - a_n}
      < \frac{\eps}{2} \\
      &\forall \eps > 0, \exists N_2: \forall n_k > k \ge N_2, \abs{x_{n_k} - l}
      < \frac{\eps}{2}.
    \end{split}
  \end{equation}

  Let $N = \max (N_1, N_2)$. Then we completes the proof by following:
  \begin{equation}
    \abs{a_n - l}
    =   \abs{a_n - x_{n_k} + x_{n_k} - l}
    \le \abs{a_n - x_{n_k}} + \abs{x_{n_k} - l}
    <   \frac{\eps}{2} + \frac{\eps}{2}
    =   \eps.
  \end{equation}
\end{proof}

\subsection{Limit Superior}

\begin{definition}
  Let $\set{a_n}$ is a bounded sequence. $\lim_{k \to \infty} \sup\set{a_n: n
      \ge k, k \in \NN}$ is a \emph{limit superior} of $\set{a_n}$ and denote
  $\varlimsup a_n$.
  % 이거 수정할 필요 있음. k를 명확하게 표현해야 함. 전부 lim으로 표현해야 할지도.
  if $a_n$ is not bounded, then we define $\varlimsup a_n = \infty$.
  \label{def:lim_sup}
\end{definition}


\begin{theorem}
  Let $\set{a_n}$ is a bounded sequence. Then,
  \begin{equation}
    \varlimsup a_n = \inf\set{\sup\set{a_n: n \ge k}: k \in \NN}.
  \end{equation}
  \label{thm:lim_sup_inf}
\end{theorem}

\begin{theorem}
  \begin{equation}
    \varlimsup a_n = \varliminf a_n = \lim a_n.
  \end{equation}
  \label{thm:lim_sup}
\end{theorem}

\subsection{Abstract}

\begin{figure}[H]
  \centering
  \begin{tikzpicture}

  \end{tikzpicture}
  \caption{Relations of theorem about sequence}
  \label{fig:}
\end{figure}


\subsection{Exercise}

1. $x_1 = 2$, $x_{n+1} = (2a_n + 3) / (a_n + 2)$.
\begin{proof}
  Claim: $x_{n+1} - (2a_n + 3) / (a_n + 2) > 0$.

  $x_1 = 2$, $x_{n+1} = (2a_n + 3) / (a_n + 2)  = a_n^2 - 3 / a_n + 2$.

  Claim: $a_n^2 > 3$.

  using induction.

  $x_1^2 = 4 > 3$.

  Assume that $x_k^2 > 3$.

  $x_{k+1}^2 - 3 = (2x_k + 3 / x_k + 2)^2 - 3 = x_k^2 - 3 / (x_k + 2)^2 > 0$.

  $a_n > x_{n+1}$ and $a_n > \sqrt{3}$.

  By MCT, $a_n$ has a limit. Let $a_n \to x$.

  $x_{n+1} \to 2x + 3 / x + 2 \implies x = \sqrt{3}$.
\end{proof}

2. Let $a_n = (-1)^n$. find $\varlimsup a_n$.
\begin{proof}
  Let $A_k = \sup{a_n: n \ge k}$. Note that $A_k = 1$. Then $\lim_{k \to \infty}
    A_k = 1$. Hence, $\varlimsup = 1$.
\end{proof}

3. find a limit superior and limit inferior of $x\frac{(-1)^n}{n}$.
\begin{proof}
  Note that $A_1 = 1/2 \ge A_2 = 1/2 \ge A_3 = 1/4 \ge \cdots \ge A_k$ and we
  have
  \begin{equation}
    \begin{split}
      A_k = 1/(k + 1), if k is odd,
      A_k = 1/k, if k is even.
    \end{split}
  \end{equation}
  Hence, $\varlimsup a_n = \lim_{k \to \infty} A_k = 0$.
\end{proof}

%******************************************************************************%
\newpage
\section{Series}

This section deals with infinite series (series for short). The following is a
definition of series.
\begin{definition}
  Let $a_n$ is a sequence. $s$ is a \emph{sum of the series} of $a_n$ if
  \begin{equation}
    s = \ser a_n = \limn\sum_{i = 1}^{n} a_i,
  \end{equation}
  and $S_n$ is a \emph{$n$-th partial sum} of $a_n$ if
  \begin{equation}
    S_n = \sum_{i = 1}^{n} a_i.
  \end{equation}
  For a series, the \emph{sum} of the series to be the limit of the partial
  sums. That is:
  \begin{equation}
    \ser a_n = \limn\sum_{n = 1}^{N}
    a_n = \limn S_n.
  \end{equation}
  If the limit exists we say the series converges, otherwise we say the series
  diverges.
  \label{def:ser}
\end{definition}

% Here is an example of series. Let $a_n = 1/n$. Then $s_n$ is not converges. To
% show that, we claim that $s_n$ is not a cauchy sequence. Let $m > n$ and $\eps =
% 1 - n/m$. Then the following shows that $s_n$ is not a cauchy sequence.
% \begin{equation}
%   \abs{s_m - s_n}
%   =   \abs{\frac{1}{n+1} + \frac{1}{n+2} + \cdots + \frac{1}{m}}
%   >   1 - \frac{n}{m}
%   =   \eps.
% \end{equation}

% Note that a converse of this theorem is not true. we already show the counter
% example, $a_n = 1/n$. This theorem use to check a series converges, since a
% limit of sequence is not $0$, then a series diverges by the theorem. It is why
% the theorem called `Divergence test of series'.

\begin{theorem}
  Let serieses $\ser a_n$, and $\ser b_n$ are
  converge. Then,
  \begin{itemize}
    \item $\ser (a_n \pm b_n) = \ser a_n \pm \ser b_n$.
    \item $\ser c \cdot a_n = c \cdot \ser a_n$.
  \end{itemize}
  \label{thm:ser_lin} % linearlity of series
\end{theorem}
\begin{proof}

\end{proof}

% Consider a seires $\ser \brk{3/n(n+1) + 1/2^n}$. By the theorem,
% \begin{equation}
%   \ser \brk{\frac{3}{n(n+1)} + \frac{1}{2^n}}
%   = 3 \ser \frac{1}{n(n+1)} + \ser \frac{1}{2^n}
%   = 3 \ser \brk{\frac{1}{n} - \frac{1}{n+1}} + 1
%   = 3 \cdot \brk{1 - \frac{1}{n+1}} + 1 = 4.
% \end{equation}

\subsection{Tests of Series}

\begin{theorem}[Divergence test of series]
  If $\ser a_n$ converges, then $\limn a_n = 0$.
  \label{thm:ser_test_div}
\end{theorem}
\begin{proof}
  Let $s = \limn s_n$. Since $a_n = s_n - s_{n-1}$,
  \begin{equation}
    \limn a_n = \limn (s_n - s_{n-1}) = \limn s_n - \limn s_{n-1} = s - s = 0,
  \end{equation}
  and we completes the proof.
\end{proof}

\begin{theorem}[Direct comparison test]
  Let $\exists N: \forall n \ge N, b_n \ge a_n > 0$. If $\ser b_n$ converges, then
  $\ser a_n$ convereges.
  \label{thm:ser_test_comp}
\end{theorem}
\begin{proof}
  Let $S_n = \sum_{i=1}^n a_i$, and $\ser b_n = \beta$. Then
  \begin{equation}
    \sumi{n} a_i \le \sumi{n} a_i + \sumi{n} b_n \le \sumi{n} a_i + \ser b_n = S_n + \beta.
  \end{equation}
  So, $S_n$ is upper bound. By MCT, $S_n$ converges. We completes the proof.
\end{proof}

\begin{theorem}[Limit comparison test]
  Let $\forall n, a_n, b_n > 0$ and $0 < c < \infty$.
  \begin{itemize}
    \item If $\limn \frac{a_n}{b_n} = c$, then $\ser a_n$ and $\ser b_n$ either
          both converge or diverge.
    \item If $\limn \frac{a_n}{b_n} = 0$ and $\ser b_n$ converges, then $\ser
            a_n$ converges.
    \item If $\limn \frac{a_n}{b_n} = \infty$ and $\ser b_n$ diverges, then
          $\ser a_n$ diverges.
  \end{itemize}
  \label{thm:ser_test_lim}
\end{theorem}
\begin{proof}
  Since $\limn \frac{a_n}{b_n} = c$,
  \begin{equation}
    \abs{\frac{a_n}{b_n} - c} < \frac{c}{2}.
  \end{equation}
  It follow that
  \begin{equation}
    -\frac{c}{2} < \frac{a_n}{b_n} - c < \frac{c}{2}
    \implies \frac{c}{2} \cdot b_n < a_n < \frac{3c}{2} \cdot b_n
    \implies \sumi{n} \frac{c}{2} \cdot b_n < \sumi{n} a_n < \sumi{n} \frac{3c}{2} \cdot b_n.
  \end{equation}
  By comparison test, $\ser a_n$ converges if $\ser b_n$ converges. If $\ser
    b_n$ diverges, then clearly $\ser a_n$ diverges. We completes the proof.
\end{proof}

\begin{theorem}[Integral test]
  Let $f$ is monotone decreasing and $f(x) \ge 0$ on $[N, \infty)$ for some $N
    \in \NN$. Then $\int_N^{\infty} f(x)$ and $\ser f(n)$ are either both converge
  or diverge.
  \label{thm:ser_test_int}
\end{theorem}
\begin{proof}
  Let $K \ge N$. Then $\exists x \in \RR: K \le x < K + 1$. Since $f$ is
  monotone decreasing, $f(k) \ge f(x) \ge f(k+1)$. It follows that
  \begin{equation}
    \begin{split}
      &f(k) = \int_k^{k+1} f(k)\,dx \ge \int_k^{k+1}f(x)\,dx \ge \int_k^{k+1}f(k+1)\,dx = f(k+1).  \\
      &\implies \sum_{K=N}^n a_{k+1} \le \int_{N}^{n+1} f(x)\,dx \le \sum_{K=N}^n a_{k}.
    \end{split}
  \end{equation}
  If $\int_{N}^{n+1} f(x)\,dx$ converges, then $\sum_{K=N}^n a_{k+1}$ conveges.
  If $\int_{N}^{n+1} f(x)\,dx$ diverges, then $\sum_{K=N}^n a_{k}$ diverges. We
  completes the proof.
\end{proof}

\begin{theorem}[p-series test]
  \begin{itemize}
    \item If $p \le 1$, then $\ser \frac{1}{n^p}$ diverges.
    \item If $p > 1$, then $\ser \frac{1}{n^p}$ converges.
  \end{itemize}
  \label{thm:ser_test_p}
\end{theorem}
\begin{proof}

\end{proof}



\subsection{Alternating Series}

\begin{definition}
  Let $b_n > 0$. A series $\ser (-1)^nb_n$ is \emph{alternating seires}.
  \label{def:ser_alt}
\end{definition}

\begin{theorem}[Alternating series test]
  If $\forall n, b_n \ge b_{n+1}$ and $\limn b_n = 0$, then $\ser (-1)^n b_n$
  converges.
  \label{thm:ser_alt_conv}
\end{theorem}
\begin{proof}
  Since $\forall n, b_n \ge b_{n+1}$, we have $\forall n, S_{2n} \ge S_{2n-2}$.
  Note that $S_{2n} \le b_1$. By MCT, $S_{2n}$ converges. Let $\limn S_{2n} =
    s$. Since $\limn b_n = 0$, we have $\limn S_{2n + 1} = \limn (S_{2n} +
    b_{2n+1}) = \limn S_{2n} + \limn b_{2n+1} = s + 0 = 0$. we completes the proof.
\end{proof}



\subsection{Absolute Convergence and Conditional Convergence}

\begin{definition}
  A series $\ser a_n$ is \emph{absolute convergence}
  if $\ser \abs{a_n}$ converges. $\ser a_n$ is \emph{conditional convergence} if
  $\ser a_n$ converges but It is not absolute convergence.
  \label{def:ser_abs}
\end{definition}

\begin{theorem}[Absolute convergence test]
  If a series $\ser a_n$ is absolute convergence, then $\ser a_n$ converges.
  \label{thm:ser_test_abs}
\end{theorem}
\begin{proof}
  Note that $\forall n, a_n \le \abs{a_n}$. Then
  \begin{equation}
    a_n \le \abs{a_n} \implies a_n + \abs{a_n} \le 2\abs{a_n} \implies
    \ser (a_n + \abs{a_n}) \le 2 \cdot \ser \abs{a_n}.
  \end{equation}
  Since $\ser \abs{a_n}$ converges, $\ser (a_n + \abs{a_n})$ converges.
  Let $\ser (a_n + \abs{a_n}) = k$. Then
  \begin{equation}
    \ser a_n = \ser (a_n + \abs{a_n} - \abs{a_n}) = k - \ser \abs{a_n}.
  \end{equation}
  Since $\ser \abs{a_n}$ converges, $\ser a_n$ converges. We completes
  the proof.
\end{proof}

\begin{theorem}[Ratio Test]
  Let $a_n > 0$ and $\limn \frac{a_{n+1}}{a_n} = l > 0$.
  \begin{itemize}
    \item If $l < 1$, then $\ser a_n$ converges.
    \item If $l > 1$, then $\ser a_n$ diverges.
  \end{itemize}
  \label{thm:ser_test_ratio}
\end{theorem}
\begin{proof}
  노트보세요.
\end{proof}

\begin{theorem}
  Let $a_n > 0$ and $\limn \sqrt[n]{a_n} = l > 0$.
  \begin{itemize}
    \item
  \end{itemize}
  \label{thm:ser_test_root}
\end{theorem}

\subsection{Exercises}
\begin{itemize}
  \item Show that $\ser \frac{1}{n}$ diverges. (This is a counter example of converse of theorem \ref{thm:ser_test_div}.)
  \item Is $\ser \frac{n^2}{5n^2 + 4}$ diverges, or converges?
  \item For $\abs{r} < 1$, $\ser r^n$ converges.
  \item Find the sum of $\ser \brk{\frac{3}{n(n+1)} + \frac{1}{2^n}}$.
  \item Is $\ser \frac{n}{3^n}$ converges?
  \item Is $\ser \frac{5n^3-3n}{n^2(n-2)(n^2+5)}$ converges?
  \item Is $\ser \frac{1}{n}$ converges? (Use an integral test)
  \item Is $\ser \frac{1 + \cos n}{4^n+2n^4}$ converges?
  \item Is $\ser \frac{\sqrt{n}+4}{n^2}$ converges?
  \item Is $\ser (-1)^{n-1}/n$ converges? (Hint: Let $b_n = 1/n$. Then $b_n$ is
        decreasing. We can use alternating series test.)
  \item $\ser \frac{(-1)^{n+1}n^2}{n^3+1}$ converges? (Hint: Let $f(x) =
          \frac{x^2}{x^3+1}$. Then $f(x)$ is decreasing function by calculus)
  \item Is $\ser \frac{\cos n}{n^2}$ converges? (Hint: Use an absolute test)
  \item Let $\ser n(4x^2)^n$ converges. Find a bound of $x$. (Hint: Use a ratio test)
  \item Is $\ser (-1)^n \frac{n^3}{3^n}$ converges? (\hint{Use a ratio test})
  \item Is $\ser (\frac{n}{n+1})^n$ converges? (\hint{Use a root test})
  \item Is $\ser \cos(\frac{\pi}{n})$ converges?
  \item Let $a_n > 0$ and $\limn n a_n \neq 0$. Is $a_n$ converges? (\hint{Use a
          limit comp test})
  \item Is $\ser \frac{n^n}{n!}$ converges? (\hint{USE A RATIO TEST IF A
          SEQUENCE HAVE FACTORIAL.})
  \item Show that $\forall x, \sum_{n=0}^{\infty} \frac{x^n}{n!}$ converges and
        $\forall x, \limn \frac{x^n}{n!} = 0$.
\end{itemize}

% 입실론 문제를 줄 때, 입실론을 고정시키고 극한값을 가지는 최소의 N을 구하라고 말할 것.
% 입실론은 움직이면 안됨.
% 예제 중심.

%******************************************************************************%
\newpage
\section{Limit of a Function}

\subsection{Limit Point}

\begin{definition}
  Let $c, \eps \in \RR, \eps > 0$. $(c-\eps, c+\eps)$ is a $\eps$-neighborhood
  of $c$. we denote this as $N_{\eps}(c)$.
  \label{def:neighbor}
\end{definition}

\begin{definition}
  $N_{\eps}$ we denote this as $N_{\eps}(c)$.
  \label{def:neighbor_delete}
\end{definition}

\begin{definition}

  \label{def:limit_point}
\end{definition}

\begin{theorem}

  \label{thm:}
\end{theorem}




\subsection{Continuous Function}

\begin{definition}
  
  \label{def:continuous_function}
\end{definition}