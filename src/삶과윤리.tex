\chapter{Life and Ethics}

수업참여 10점: 자기소개 및 수업신청 이유 등
한글 워드 1장, 사진 전화번호 등도 첨부.

중간 30점 - 3페이지 글자크기 10 줄간격 160 자유로운 에세이 표절 gpt 금지.
기말 40점 - 4페이지

이름 학과 수업신청 이유 칼라 추출.
토론결과 발표. 한 팀.

조장 30초. 스트레칭?.

\section{Introduction}

\begin{itemize}
  \item 공리주의: 전체의 효율성을 강조. 정의를 사회 전체의 행복을 극대화하는
        데에서 찾는다.
  \item 자유주의: (자유지상주의(강경한 자유주의)) 개인의 선택의 자율 보장하는
        것을 최우선으로 여김. 권리 > 선.
  \item 칸트주의: 엄격한 자유주의. 도덕 = 정언 명령에 따른 자유로운 행동. 자유
        아래 의무를 인식하고 행동해야 함.
  \item 평등자유주의: 기본적 자유는 침해하지 않는 선에서 평등원칙과 차등원칙을 강조.
        자유주의에서 사회주의를 약간 가져옴.
  \item 미덕주의: 공동체의 미덕을 구현한 사람에게 권력을 주는 것이 정의.
        공동선의 본질을 파악한 사람에게 최곡 공직과 영예가 돌아가야 함.
\end{itemize}

트롤리 문제: 윤리적 딜레마
플라톤의 동굴: 우리가 알고 있는게 정말 맞는 것인가?

확정 편향하는 이유는 체력적으로 편하게 하기 위함.

토론 주제: 동굴 밖에 나온 사람은 동굴 안의 사람을 계몽시켜야 한다. 그것이 철학자가 할 일이다.
이것에 대해 어떻게 생각하는가?

%******************************************************************************%
\section{Ethical Dilemma}

\begin{exercise}
  임수 수행을 위해 목표 지점에 미사일을 발사하려고 하는데, 목표 지점 근처에 어린
  아이가 있다. 만약 당신이라면 어떻게 할 것인가? 그리고 그렇게 생각한 이유는
  무엇인가?
\end{exercise}

\begin{solution}
  이윤기: (바로 쏘는)쏜다. 이유: 대를 위한 소의 희생.
  -> 반박: 대가 어느 쪽인지 알 수 없다. 민간인을 죽인게 여론 상 불리할 수 있다.
  최이안: 기회를 주고, 실패 직전에 쏜다. 이유: 조금 기다린다고 하더라도 바로 실패하지 않는다.
  -> 반박: 기회를 준다는 게 실패로 이어질 수 있다. 쉽지 않음.
  김재학: 쏜다. 이유: 테러리스트들이 어떤 사람들을 죽일지 모른다. 빨리 죽이는 게 맞다.
  -> 반박: 대가 어느 쪽인지 알 수 없다. 민간인을 죽인게 여론 상 불리할 수 있다.
  김용우: 기회 주고 쏜다. 주요 목적이 테러리스트 죽이는 것. 윤리적 문제 고려할 때 기다리고 쏜다.
  김동현: 안쏜다. 이유: 지금 안쏘더라도 기회는 온다. 굳이 민간인이 피해를 입는 상황에서 쏠 필요가 없다.
  -> 반박: 기회가 언제 올 지 모른다. 기회가 오기 전에 죽을 수 있음.
\end{solution}

\begin{exercise}
  민간인을 어떻게 할까? 1. 놔두고 올라가는 것(실제로 선택함). 2. 묶고 운에 맡기는 것. 3. 죽이는 것.
\end{exercise}

\begin{solution}
  김동현: 3. 죽인다. 민간인이 아닐 수 있음. 놔뒀을 때 임무가 실패할 확률이 높다.
  김용우: 3. 죽인다. 신원확인이 안됨. 놔뒀을 때 임무가 실패할 확률이 높다.
  -> 반박: 민간인일 수 있다. 라는 점이 문제가 된다. 죽인다고 해서 임무가 성공할 수 있나?
  죽인다고 임무 성공 보장이 없다.
  최이안: 2, 3: 첫 번째랑 다른점: 지금 상황은 잘못되면 우리가 죽는다. 안죽는 상황을 만든다.
  -> 반박: 민간인일 수 있다. 라는 점이 문제가 된다. 죽인다고 해서 임무가 성공할 수 있나?
  죽인다고 임무 성공 보장이 없다.
  김재학: 2: 묶어둔다. 이유: 직접 죽이는 건 죄책감이 든다. 임무는 성공해야 함. 임무가 성공을 한다면 굳이 죽일 이유가 없다.
  -> 반박: 결과가 달라지지 않는다. 리스크는 있음. 어중간한 느낌이다.
  이윤기: 3: 죽인다. 이유: 불안하다. 살아있으면, 신경쓰여서 임무에 집중이 안될 수 있다.
  -> 반박: 죽여도 죄책감에 집중이 안될 수 있다.
\end{solution}

%******************************************************************************%
\section{Utilitarianism}

공리주의적으로 생각할 때 총 합이 최대의 효율을 내야 함.

피자 5조각과 아이 5명. 피자에 대한 쾌락의 양이 다르다.

[이름] [1조각에 대한 쾌락] [2조각에 대한 쾌락]
이승우 11 2
이미영 9 6
한송희 4 1
김덕룡 7 3
박서현 10 5

공리주의적으로 볼 때, 송희는 안주고 미영이에게 2조각을 준다. 나머지는 1조각씩 준다.

공리적이려면, 일단 합리적, 이타적이어야 한다. 그래야 작동한다.

[유죄인가? 무죄인가? 유죄라면 얼만큼?]
시신을 버리지 않으면 우리는 살 수 있다.
제비뽑고 죽을 사람을 선택.

김동현: 무죄: 정당방위, 어쩔 수 없음.
이윤기: 무죄: 죄를 따질 문제가 아니다?
김재학: 무죄: 동의 하에 죽었으니 이건 자연사다.
최이안: 유죄: 서로가 어쩔 수 없는 동등한 상황이었어. 그래서 당위성을 부여할 수 없다.

공리주의의 이익/쾌락 원리
감정에는 쾌락과 고통 2가지가 있다. (모든 욕구를 쾌락과 고통으로 치환 가능하다.)

공리주의는 생명을 돈으로 환산가능ㅎ마. 거부감은 극복해야 할 충동적 감정임.

\subsection{John Stut Mile}

존 스튜어트 밀은 공리주의 원칙에 인간적인 접근을 시도했다. 밀은 경험주의 인식론,
공리주의 윤리학을 가진 자유주의자. 자유주의적 정치경제 사상을 가족 있었으나,
사회주의적 요소도 일부 가지고 있다. 공리주의에 공상적 사회주의와 낭만주의를 가미했다.
밀은 자유를 포기하지도, 타인의 자유를 침해하지도 않았던 자유주의자.

밀은 양뿐만 아니라 질적 쾌락도 고려해야 한다고 한다. 쾌락과 고통이 전부이나, 더 가치있는 쾌락이 존재한다고 주장
배부른 돼지와 배고픈 소크라테스. ?

무엇이 더 나은 쾌락인가? 1. 세계격투기, 2. 셰익스피어 햄릿, 3. 사망토론
햄릿이 가장 고급스러운 쾌락이다. 이유는 영혼의 고급능력을 이끌어내기 때문이다.

질적으로 다른 쾌락이 존재한다. 아니다.

김용우: 아니다. 이유: 질적 쾌락의 기준이 애매하다.
최이안: 존재한다. 이유: 운동의 쾌락, 누워서 보는 쾌락은 질이 더 낮다.
김재학: 존재하고. 이유: 지속가능한 쾌락이 질적으로 나은 쾌락이다.
김동현: 아니다. 이유: 질적 쾌락과 양적 쾌락을 나눌 수 없다.

테러리스트를 고문해서 핵을 멈춰야 하는가?

최이안: 테러리스트까지는 고문 가능. 행동에 대한 대가를 치루고 있는거니까. 근데
주변인들까지? 이건 또 다른 문제다. 책임의 무게에 따라 선택할 수 있다.
김재학: 전부 고문함. 많은 사람들을 살리는 것이 가장 중요하다.
김용우: 테러리스트까지는 고문 가능. 행동에 대한 대가를 치루고 있는거니까. 근데
주변인들까지? 이건 또 다른 문제다.
김동현: 전부 고문함. 많은 사람들을 살리는 것이 가장 중요하다.

%******************************************************************************%
\section{Liberalism}

자유주의는 온정주의, 도덕법제화, 소득과 부의 재분배를 거부한다. 개인의 신체나 목숨에 간섭하는 법에
반대하고, 제도나 법을 통한 미덕을 강조하는 것을 반대하고, 과도한 세금 징수를 거부한다.

최해갑 가족의 행위는 자유를 향한 투쟁의 과정인가? 아니면 공동체로부터 이탈하고자 하는 윤리적 행위인가?
\begin{itemize}
  \item 이윤기: 반 윤리적 행위: 국가가 그 사람에게 주는 혜택이 있는데, 그만큼
        의무를 다해야 한다. 의무를 다하지 않고 있는 것이다.
  \item 김용우: 반 윤리적 행위: 이유가 있어서 철거는 하는건데, 공동체적으로 볼
        때, 평화와 질서를 위한 일인데 자기만을 위해서 하는 건 반 윤리적 행위이다.
  \item  김재학: 자유를 행한 투쟁: 남한테 피해를 주지 않았다. 그냥 내가 여기에
        살고 있는 것 뿐이다.
  \item 최이안: 반 윤리적 행위: 국가가 그 사람에게 주는 혜택이 있는데, 그만큼
        의무를 다해야 한다. 의무를 다하지 않고 있는 것이다.
  \item 김동현: 반 윤리적 행위: 비도덕적인 것 같다. 남들 다하는데 나만 안하면
        이건 남한테 피해를 주는 것 같다.
\end{itemize}

하이에크에 따르면, 인간의 이성은 불완전하며, 이러한 무지야말로 인간에게 자유가
필요한 근거이다. 구성주의적 합리주의는 이성의 힘을 과신하여, 자유를 파괴하는
오류를 범했다.

로버트 노직의 자유지상주의. 존 롤스에 대항하는 자유지상주의 정치 철학적 논의. 노직 정치철학의 특성은
존 로크의 자연상태 논의를 빌려와 정치철학을 언급. 과세는 강제노동과 같다. 야경국가론, 최소국가론 주장.
노직은 롤즈의 임의성-우연성 개념에 비판적 입장 견지.

자유지상주의의 소유 논리. 개인의 소유권은 능력의 소유 및 능력에 따른 결과물의 소유를 의미.

\subsection{Tax}

프랑스 배우 제라드 드 파르디외 세금 망명. 프랑스에서 연간 100만 유로 이상의
소득에 대해 내년부터 최대 75퍼센트의 세금을 부과하는 정책을 강화. 루이비통
회장은 벨기에 국적을 신청했으며, 클라비에르는 영국행을 발표했다.

찬성 : 이득이면 무조건 국적을 바꾼다. 국가가 비도덕적인데 내가 도덕을 따질 이유가 없다.
찬성 : 50퍼센트는 너무 세다.
전체 찬성.

\subsection{Euthanasia}

안락사 반대 : 허용하면 향 후에 자식들이 부모들에게 눈치를 준다. (사회적인 문제가 생길 수 있다.)

안락사 찬성 : 아픔을 느끼면서 죽는 것 보다 고통없이 죽는 게 더 낫다.

안락사 반대 : 합법화되면, 죽음에 대해 가볍게 생각 할 수 있다. 충동적으로 안락사를 결정했을 때 돌이킬 수 없다.

안락사 찬성 : 아픔을 느끼면서 죽는 것 보다 고통없이 죽는 게 더 낫다.

\subsection{Organ Trafficking}

장기매매
반대 : 장기를 사는 사람이 있는데, 사람 살리려고 하는데 거래를 한다? 돈 없는 사람은 이식을 못 받음.
장기기증은 가능.
찬성 : 파는 사람 마음이다. 돈 받고 주면 윈윈이다.
찬성 : 장기부족으로 치료 못받는 사람들을 위한 것이 될 수 있다. 파는 사람 마음이다.

\subsection{Cannibalism}

합의 하에 하는 식인 :

반대 : 두 사람이 정상적인 판단? 이게 합의를 한 게 맞나?
반대 : 사회가 합의를 안했어.
반대 : 합의에 대한 불확실성.
반대 : 시체가 비인륜적으로 사용될 것이다.

\subsection{Conscription}

자유주의는 징병제를 반대하고, 공리주의자는 모병제를 옹호한다.

모병제와 징병제에 대한 토론.
여성징병제에 대한 토론.

모병제: 군인은 로봇으로 대체하면 나머지는 로봇을 관리하는 사람들로 쓰면 됨.
강제적으로 긴 시간을 뺏어가는건 도덕적이지 않다.
징병제가 공리주의적으로 볼 때 효율적이지 않을 수 있다. 다른 곳에서 효율적으로 쓸 수 있음.
징병제: 휴전 중임. 위험을 느껴야 하는 상황이기 때문에. ㅋㅋㅋㅋㅋㅋ

여성징병
찬: 같은 경험을 함으로써 사회의 단합력이 올라갈 수 있다.
반: 여성을 징병한다고 군대의 효율이 올라갈지 미지수이다.




대리more에 대하여...

출산 X => 근데 딸은 키우고 싶어 => 대리more

관계는 돈을 내고 하는게 아닌 가족관계 내에서 위해서 하는.

찬성: 아무도 손해를 보고 있지 않음, 문제가 될 게 없음.
반대:  도덕적으로 문제가 있다, 건강하지 않은 아이를 낳을 수 있다,
핏줄이 중요하다.


출산능력을 사고 파는 것은 그 가치를 비하하는 것.
대리 출산은 여성과 아이를 상품화.

%******************************************************************************%
\newpage
\section{Kantianism}

칸트의 자유주의, 줄여서 칸트주의는 엄격하고 금욕적인 자유주의이다. 공리주의와
미덕주의를 비판하고, 인간을 이성적, 자율적, 존엄적, 목적적 존재로 본다.

칸트의 도덕기준은 선의지(양심)이다. 욕구, 기호, 신의 권의(예: 십계명) 등은
도덕기준이 아니다. 선의지는 의무동기이다. 의무동기라야 정의롭다. 경향성 동기는
정의롭지 않다. 이는 옳은 행동으로 보이더라도 이익을 먼저 생각하는 동기이기
때문이다.

칸트는 이성은 선의지를 결정짓고 법칙을 만든다고 한다. 칸트의 이성은 선한 의지

정언명령은 모든 사람이 무조건적으로 따라야 하는 규범으로서의 명령이다. 하지만
가언명령은 조건적, 상대적 명령으로 배후의 목적, 의도가 있다. 정언명령은 '너의
의지의 준칙이 항상 동시에 보편적 법칙 수립의 원리로서 타당할 수 있어야 한다.

할 수 있다고 봐요. 납치범의 요구가 허들이 낮아. 사람을 살릴 수 있는데 안막는건
비 도덕적이다. 도덕적으로 보면 하는게 맞다. 요구가 낮으면 안살려줄 것 같아.
안한다. : 대통령의 체면이 더 중요하다. 공주는 다시 낳으면 된다.


\section{}
6천만원의 빚을 4명의 형제자매가 값아야 하는 상황.
형: 의사 독신 1억.
제: 실직 남편 자녀 계약직 2천.
자: 아내 딸두명 중소기업 4천.
매: 직업 있는 아내 자녀 2명 공무원 ??.

1. 4명의 신원을 모르는 상태에서 객관적으로 정해보자.
n빵한다:
이유는 모르겠지만 n빵이 맞는 것 같다. 제일 공평한 것 같다.
사전에 제일 부자인 사람이 내기로 약속한다:
이후 3천 2천 1천 0 순으로 낸다.

2. 역할 분담을 한 상황에서 주관적으로 정해보자.
막내 연소득 2천: 난 안낸다: 천만원은 내겠다.
장녀 연소득 1억: 최대 3천까지는 내겠다.
장남 연소득 4천: 2천까지는 내겠다.
차남 연소득 8천: 1천5백까지 내겠다.

분배 정의와 관련된 4가지 이론과 정의 실현
\begin{itemize}
  \item 봉건/카스트 제도
  \item 자유지상주의: 경쟁을 통해 많이 벌면 많이 버는 거고, 적게 벌면 적게 버는 것.
  \item 능력주의: 재능도 공동 자산으로 나눠줘야 하는 것.
  \item 평등(자유)주의: 우연적으로 얻은 것을 나눠야 한다.
\end{itemize}

도와줄 필요 없다: 
  도와주면 일을 더 열심히 안할 것. 저만큼만 하고 살게 될 것.
  도와주면 스스로 살 의지를 잃어버리게 된다.
  혜택의 목적: 스스로 살도록 만드는게 목적. 생산적이지 않은 사람한테 돈 줄 필요 없음.
  술, 담배하는 사람한테 돈주기 싫음. 

소수집단에 대하여.

소수집단우대정책에 대해 어떻게 생각하는가?

1. 예전에는 정말 필요한 정책이지었지만, 지금은 필요 없는 정책.
2. 차별이 있고, 그 대응이 필요할 때 사용.
3. 차별을 당했을 때 펼쳐야 하는 정책이다. 소수가 중요한 게 아닌, 차별을 당하고 있는 그 상황이 중요하다.

1. 능력대로 뽑아야 한다.
2. 그게 평등한 사회이다.

소수집단우대정책이 존재할 때 어떤 문제가 있는가?

1. 소수집단 학생들의 자존심 훼손.
      -> 차별당하는 학생들의 자존심은 고려 안하나?
2. 인종 문제 노출.
      -> 인종 문제를 모른체하고 있자는 얘기인가?
3. 역차별자의 분노.
      -> 차별자의 분노는 고려 안하나?

소수집단우대정책을 펼쳤을 때, 역차별당한 사람과 

오히려 소수집단우대정책이 없을 때 더 큰 문제가 많을 수 있다.