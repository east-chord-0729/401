\chapter{삶과 윤리}

수업참여 10점: 자기소개 및 수업신청 이유 등
한글 워드 1장, 사진 전화번호 등도 첨부.

중간 30점 - 3페이지 글자크기 10 줄간격 160 자유로운 에세이 표절 gpt 금지.
기말 40점 - 4페이지

이름 학과 수업신청 이유 칼라 추출.
토론결과 발표. 한 팀.

조장 30초. 스트레칭?.

\section{윤리학의 핵심 주제들}

\begin{itemize}
  \item 공리주의: 전체의 효율성을 강조. 정의를 사회 전체의 행복을 극대화하는
        데에서 찾는다.
  \item 자유주의: (자유지상주의(강경한 자유주의)) 개인의 선택의 자율 보장하는
        것을 최우선으로 여김. 권리 > 선.
  \item 칸트주의: 엄격한 자유주의. 도덕 = 정언 명령에 따른 자유로운 행동. 자유
        아래 의무를 인식하고 행동해야 함.
  \item 평등자유주의: 기본적 자유는 침해하지 않는 선에서 평등원칙과 차등원칙을 강조.
        자유주의에서 사회주의를 약간 가져옴.
  \item 미덕주의: 공동체의 미덕을 구현한 사람에게 권력을 주는 것이 정의.
        공동선의 본질을 파악한 사람에게 최곡 공직과 영예가 돌아가야 함.
\end{itemize}

트롤리 문제: 윤리적 딜레마
플라톤의 동굴: 우리가 알고 있는게 정말 맞는 것인가?

확정 편향하는 이유는 체력적으로 편하게 하기 위함.

토론 주제: 동굴 밖에 나온 사람은 동굴 안의 사람을 계몽시켜야 한다. 그것이 철학자가 할 일이다.
이것에 대해 어떻게 생각하는가?

\section{윤리적 딜레마}

\begin{exercise}
  임수 수행을 위해 목표 지점에 미사일을 발사하려고 하는데, 목표 지점 근처에 어린
  아이가 있다. 만약 당신이라면 어떻게 할 것인가? 그리고 그렇게 생각한 이유는
  무엇인가?
\end{exercise}

\begin{solution}
  이윤기: (바로 쏘는)쏜다. 이유: 대를 위한 소의 희생.
  -> 반박: 대가 어느 쪽인지 알 수 없다. 민간인을 죽인게 여론 상 불리할 수 있다.
  최이안: 기회를 주고, 실패 직전에 쏜다. 이유: 조금 기다린다고 하더라도 바로 실패하지 않는다.
  -> 반박: 기회를 준다는 게 실패로 이어질 수 있다. 쉽지 않음.
  김재학: 쏜다. 이유: 테러리스트들이 어떤 사람들을 죽일지 모른다. 빨리 죽이는 게 맞다.
  -> 반박: 대가 어느 쪽인지 알 수 없다. 민간인을 죽인게 여론 상 불리할 수 있다.
  김용우: 기회 주고 쏜다. 주요 목적이 테러리스트 죽이는 것. 윤리적 문제 고려할 때 기다리고 쏜다.
  김동현: 안쏜다. 이유: 지금 안쏘더라도 기회는 온다. 굳이 민간인이 피해를 입는 상황에서 쏠 필요가 없다.
  -> 반박: 기회가 언제 올 지 모른다. 기회가 오기 전에 죽을 수 있음.
\end{solution}

\begin{exercise}
  민간인을 어떻게 할까? 1. 놔두고 올라가는 것(실제로 선택함). 2. 묶고 운에 맡기는 것. 3. 죽이는 것.
\end{exercise}

\begin{solution}
  김동현: 3. 죽인다. 민간인이 아닐 수 있음. 놔뒀을 때 임무가 실패할 확률이 높다.
  김용우: 3. 죽인다. 신원확인이 안됨. 놔뒀을 때 임무가 실패할 확률이 높다.
  -> 반박: 민간인일 수 있다. 라는 점이 문제가 된다. 죽인다고 해서 임무가 성공할 수 있나?
          죽인다고 임무 성공 보장이 없다.
  최이안: 2, 3: 첫 번째랑 다른점: 지금 상황은 잘못되면 우리가 죽는다. 안죽는 상황을 만든다.
  -> 반박: 민간인일 수 있다. 라는 점이 문제가 된다. 죽인다고 해서 임무가 성공할 수 있나?
          죽인다고 임무 성공 보장이 없다.
  김재학: 2: 묶어둔다. 이유: 직접 죽이는 건 죄책감이 든다. 임무는 성공해야 함. 임무가 성공을 한다면 굳이 죽일 이유가 없다.
  -> 반박: 결과가 달라지지 않는다. 리스크는 있음. 어중간한 느낌이다.
  이윤기: 3: 죽인다. 이유: 불안하다. 살아있으면, 신경쓰여서 임무에 집중이 안될 수 있다.
  -> 반박: 죽여도 죄책감에 집중이 안될 수 있다. 
\end{solution}