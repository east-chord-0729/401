\chapter{리더십챌린지}

리더와 팔로워.

중간 30 기말 30 과제 10 출석 20 수업참여 10(기본 7 + 플러스 요인 3(일대일 코칭 참가 등))

리더 vs 관리자.
리더. 가장 중요하게 해야 할 일이 무엇인지에 대한 합의를 이끌어 내는데 초점.
관리자. 구성원들의 성과를 향상시키는데 초점.

리더십과 성과의 관계.

성과를 높이는 것: 생산성, 결근/이직, 일탈적 직장행동(<->)조직시민행동, 조직몰입(여기서 일하고 싶어) 등등
-> 이것들을 높이는(혹은 낮추는) 원인?
= 사원의 태도/행동 (<->) 리더의 태도/행동
-> 리터의 특성/기술이 좋아야함.

리더십을 어떤 관점으로 분석할까? 
특성(내면, 타고난 재능), 행동(나타나는 행동), 권력-영향력(행동이 미치는 영향력, 사원 간의 관계), 
상황(주변 환경이 리더를 만든다.), 통합(전부 다 ㅋ)

\section{리더십의 정의}
리더십의 정의는 합의된 것이 없다. 그래도 말하자면, 리더십이란 다음 중 하나로 정의할 수 있다. 
보통 리더십에서 목표달성과 영향력은 리더십의 정의에 인정한다.
\begin{itemize}
  \item 목표달성을 위한 활동에 영향을 미치는 과정.
  \item 다른 사람들에 의해서 리더라고 인정받는 과정.
  \item 영향력 행사 과정
  \item 변화를 촉발하는 능력.
\end{itemize}

연도별 리더십: 리더와 팔로워 사잉에서 일어나는 현상에서, 관계나 행동의 패턴의 흐름.
수평적이고 민주적으로 바뀌어감.

공식적 리더, 비공식적 리더.
->공적으로 리더의 역할을 받았는가?
비공식적 리더는 뭔데?

직접적 리더, 간접적 리더
-> 직접적: 회사 상사, 교수, 동아리 회장 등. 간접적: 존경하는 사람?, 저명한 학자, 시민 운동가

\section{권력-영향력 관점에서의 리더십}

대상: 집단/개인

def. 권력이란? 대상A이 다른 대상B의 행동에 영향을 미치기 위해 가지고 있어야 하는 능력.

def. 의존성이란? B가 필요로 하는 것이 A가 소유하고 있을 경우. A에 대한 B의 관계.
B의 의존성이 커질 수록, A는 더 큰 권력을 얻음.

공식적 권력. -> 지위에 의해 형성.
  1. 보성을 줄 수 있기에 생기는 권력.
  2. 벌을 줄 수 있기에 생기는 권력.
  3. 규정, 근거에 의해 생기는 권력. (?)

개인적 권력. -> 개인의 매력, 특성에 생김
  1. 전문성이 뛰어나서 생김. 막 뭔가 배우고 싶을 때.
  2. 인간적 매력에서 생겨남. (준거적 권력)

권력을 행사하기 위한 행동 패턴은 여러가지가 있다. 이중 가장 효과 높은 세 가지는 다음과 같다. 
참고로 압력에 의한 권력행사는 효과가 낮다.
  1. 이성적 설득.
  2. 대상의 감정적 몰입을 이끌어 냄. 대상의 가치에 호소.
  3. 대상이 의사결정에 참여시키도록 함.

부드러운 전술을 먼저 사용하는 것이 효과적임.
정치적 기술(사람 간의 커뮤니티 형성 비슷한 거)이 권력 전술의 성패를 결정함.
조직의 문화에 따라 선택하는 권력 전술이 달라짐.

대표적인 영향력 발생과정
1. 수단에 의한 복종 : 보상 혹은 처벌 때문에 행동을 따름.
2. 내면화 : 주어진 제안이 자신의 가치와 신념에 빗대어 볼 때 옳다고 생각하고 행동을 따름.
3. 개인적 동일시 : 존경에 의한 행동 모방.

영향력 행사의 결과
1. 몰입 : 제안을 효과적으로 실행하기 위해 노력함.
2. 복종 : 제안을 응하지만 열성적이진 않음. 최소한의 노력만 함.
3. 저항 : 제안을 응하지 않음.
뭐, 딱봐도 1 > 2 > 3번이죠 성과의 크기가.

리더의 권력 -a- 리더의 영향력 행사 -b- 행사의 결과 -c- 리더의 권력
a: 권력에 따른 행동의 차이
b: 권력과 행동이 미치는 결과
c: 권력에 따른 태도

적절한 권력 전술을 선택하고 행사.
지위에 의존하지말고 전문가, 준거적 권력을 개발.