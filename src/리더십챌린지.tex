\chapter{Leadership Challenge}

리더와 팔로워.

중간 30 기말 30 과제 10 출석 20 수업참여 10(기본 7 + 플러스 요인 3(일대일 코칭 참가 등))

리더 vs 관리자.
리더. 가장 중요하게 해야 할 일이 무엇인지에 대한 합의를 이끌어 내는데 초점.
관리자. 구성원들의 성과를 향상시키는데 초점.

리더십과 성과의 관계.

성과를 높이는 것: 생산성, 결근/이직, 일탈적 직장행동(<->)조직시민행동, 조직몰입(여기서 일하고 싶어) 등등
-> 이것들을 높이는(혹은 낮추는) 원인?
= 사원의 태도/행동 (<->) 리더의 태도/행동
-> 리터의 특성/기술이 좋아야함.

리더십을 어떤 관점으로 분석할까? 
특성(내면, 타고난 재능), 행동(나타나는 행동), 권력-영향력(행동이 미치는 영향력, 사원 간의 관계), 
상황(주변 환경이 리더를 만든다.), 통합(전부 다 ㅋ)

%******************************************************************************%
\section{Definition of Leadership}
리더십의 정의는 합의된 것이 없다. 그래도 말하자면, 리더십이란 다음 중 하나로 정의할 수 있다. 
보통 리더십에서 목표달성과 영향력은 리더십의 정의에 인정한다.
\begin{itemize}
  \item 목표달성을 위한 활동에 영향을 미치는 과정.
  \item 다른 사람들에 의해서 리더라고 인정받는 과정.
  \item 영향력 행사 과정
  \item 변화를 촉발하는 능력.
\end{itemize}

연도별 리더십: 리더와 팔로워 사잉에서 일어나는 현상에서, 관계나 행동의 패턴의 흐름.
수평적이고 민주적으로 바뀌어감.

공식적 리더, 비공식적 리더.
->공적으로 리더의 역할을 받았는가?
비공식적 리더는 뭔데?

직접적 리더, 간접적 리더
-> 직접적: 회사 상사, 교수, 동아리 회장 등. 간접적: 존경하는 사람?, 저명한 학자, 시민 운동가

%******************************************************************************%
\section{Power-Influence Leadership}

대상: 집단/개인

def. 권력이란? 대상A이 다른 대상B의 행동에 영향을 미치기 위해 가지고 있어야 하는 능력.

def. 의존성이란? B가 필요로 하는 것이 A가 소유하고 있을 경우. A에 대한 B의 관계.
B의 의존성이 커질 수록, A는 더 큰 권력을 얻음.

공식적 권력. -> 지위에 의해 형성.
  1. 보성을 줄 수 있기에 생기는 권력.
  2. 벌을 줄 수 있기에 생기는 권력.
  3. 규정, 근거에 의해 생기는 권력. (?)

개인적 권력. -> 개인의 매력, 특성에 생김
  1. 전문성이 뛰어나서 생김. 막 뭔가 배우고 싶을 때.
  2. 인간적 매력에서 생겨남. (준거적 권력)

권력을 행사하기 위한 행동 패턴은 여러가지가 있다. 이중 가장 효과 높은 세 가지는 다음과 같다. 
참고로 압력에 의한 권력행사는 효과가 낮다.
  1. 이성적 설득.
  2. 대상의 감정적 몰입을 이끌어 냄. 대상의 가치에 호소.
  3. 대상이 의사결정에 참여시키도록 함.

부드러운 전술을 먼저 사용하는 것이 효과적임.
정치적 기술(사람 간의 커뮤니티 형성 비슷한 거)이 권력 전술의 성패를 결정함.
조직의 문화에 따라 선택하는 권력 전술이 달라짐.

대표적인 영향력 발생과정
1. 수단에 의한 복종 : 보상 혹은 처벌 때문에 행동을 따름.
2. 내면화 : 주어진 제안이 자신의 가치와 신념에 빗대어 볼 때 옳다고 생각하고 행동을 따름.
3. 개인적 동일시 : 존경에 의한 행동 모방.

영향력 행사의 결과
1. 몰입 : 제안을 효과적으로 실행하기 위해 노력함.
2. 복종 : 제안을 응하지만 열성적이진 않음. 최소한의 노력만 함.
3. 저항 : 제안을 응하지 않음.
뭐, 딱봐도 1 > 2 > 3번이죠 성과의 크기가.

리더의 권력 -a- 리더의 영향력 행사 -b- 행사의 결과 -c- 리더의 권력
a: 권력에 따른 행동의 차이
b: 권력과 행동이 미치는 결과
c: 권력에 따른 태도

적절한 권력 전술을 선택하고 행사.
지위에 의존하지말고 전문가, 준거적 권력을 개발.

\section{Ethical, Sevent and Authentic Leadership}

가치관이란 옳고 그름의 판단 기준이다. 윤리적 가치란 '대인관계'에서 행동이
허용되는 기준이다. 리더십 연구에서 윤리는 경영을 하면서 어떤 행동까지
허용되는가의 문제이다.

황금률, 신문테스트: 내 행동이 신문에 난다면.

진성 리더십이란 리더의 자기인식, 내면화된 윤리적 신념 정보의 균형 잡힌 처리,
부하와의 곤계에서 관계적 투명성의 유지, 긍정적 자기개발을 기반으로 긍정적 윤리적
풍토와 긍정적 심리적 능력을 가화하는 리더 행동의 유형이다. 진성 리더란 자신에
대해 잘 이해하고 타인과 개방적인 의사소통을 하며 자신의 가치와 신념에 따라
행동하는 사람이다.

\section{Self-Super Leadership}

슈퍼 리더(Super Leader)는 셀프 리더십에 초점을 맞춘다. 셀프 리더십이란 자기
자신을 이끄는 리더십을 말한다. 이들은 가치를 서로 공유함으로써 권력을 행사한다.
너와 내가 중요하게 여기는 것이 똑같으니, 같이 해보자는 말이다. 이들의 전형적인
행동은 셀프리더가 되고, 팔로워들을 셀프리더가 되도록 만드는 것이다. 자기목표
설정을 독려하고, 긍정적 사고방식을 창조하는 등 셀프리더십 문화를 창달한다.

\subsection{Notion}

셀프리더십이란 개인이 자기 자신에게 영향을 미치는 지속적인 과정이다.
셀프리더십이란 자기방향 설정과 자기 동기부여를 위해 설계된 행동과 인지전략의
총체이다. 다른 사람들을 셀프리더로 만드는 리더십은 수퍼리더십이라고 한다.

셀프리더십은 사회적 학습 이론과 자기 결정 이론을 기반으로 한다. 사회적 학습
이론이란 사회적으로 옳고 그름을 판단하고 학습하여 자신의 행동의 결정을 내리는
이론이다.(내생각임) 그리고 자기 결정 이론이란 자신의 행동을 스스로 통제할 수
있을 때 동기부여 되는 이론이다. 개인의 행동을 통제하는 요인에는 역량감, 자율성,
연대감이 있다.


\subsection{?}

셀프리더십은 6가지 요인이 있다. 자기관찰은 자기를 관찰하여 변화의 포인트를
발견한다. 힌트전략은 중요한 것을 기억해두기 위해 도구를 활용하는 것이다.
자기목표설정은 성취를 위한 실행계획을 설정하는 것이다. 자기보상은 스스로에게
보상하는 것이며, 자기벌칙은 스스로에게 비판하는 것이다. 연습은 필요한 행동을
반복하여 변화시키는 것이다.

수퍼리더십은 추종자들이 자기 자신을 리드할 수 있는 역량과 기술을 갖도록 하는
리더십이다.수퍼리더에 이르는 7단계 모델은 다음과 같다. 

1. 셀프리더 되기 2. 셀프리더십 모델 되기. 3. 자기목표 설정 독려. 4. 긍정적 사고 패턴 창조. 
5. 보상과 건설적 피드백을 통한 개발. 6. 셀프리딩 팀워크 촉진. 7. 셀프리더십 문화의 창달.

%******************************************************************************%
\section{Followership}

지금까지는 리더십에 대해 알아보았다. 이 절에서는 리더가 아닌 팔로워로써 가져야
할 능력, 팔로워십(Followership)에 대해 설명한다. 

\subsection*{background}
리더와 팔로워간의 권력양상은 다양하게 나타난다. 리더가 고통받으며 일을 할 때,
팔로워는 리더에게 도움을 주는 것이다. 효과적인 리더십을 위해서는 팔로워의 기여가
필요하다.

켈리(1992)는 조직성과에 더 크게 기여하는 것은 리더가 아닌 팔로워들이며, 따라서
팔로워들을 키우기 위해 더 노력해야 한다고 주장한다.\footnote{reference 추가}
켈리는 팔로워의 자세를 수동적인가 적극적인가로 나누고, 

%******************************************************************************%
\newpage
\section{Behavioral Leadership Theory}

이번 절에서는 리더십의 행동이론에 대해 설명한다. 행동 -> 스킬 -> 스타일.

\subsection*{behavior}

리더십 행동이론이란 특정의 행동들이 리더와 리더가 아닌 사람을 구분한다는 이론이다. 리더의 행동에는 크게 
과업중심적 행동, 관계중심적 행동, 그리고 변화중심적 행동이 있다.
\begin{itemize}
  \item 과업중심적(Task oriented) 행동
  \item 관계중심적(Relationship oriented) 행동
  \item 변화중심적(Change oriented) 행동
\end{itemize}

1945년 오하이오주립대학(OSU) 연구는 바람직한 리더의 행동을 배려와 구조주도의 두
가지 행동으로 보았다. 배려 행동은 리더가 추종자들을 따듯하고 인간적으로 대해주는
행위, 구조주도 행동은 업무중심, 과업중심, 목표중심 행위를 의미한다.

블레이크와 무튼은 리더의 행동을 

\subsection*{skill}

리더는 여러 상황들을 효과적으로 극복할 수 있도록 다양한 리더십 기술을 가지고 있어야 한다.
리더십 스킬 유형은 다음과 같다.
\begin{itemize}
  \item 자신의 성장, 절제를 위한 스킬
  \item 관계의 형성, 유지, 발전을 위한 스킬
  \item 미래의 모습을 성취하기 위한 스킬
  \item 변화와 문제에 대응하는 스킬
  \item 조직의 역동적 시너지 창출 스킬
  \item 과업의 효과적 수행을 위한 스킬
\end{itemize}

\subsection*{style}

리더십 스타일이란 리더십을 발휘할 때 나타나는 행동의 패턴이다.

%******************************************************************************%
\newpage
\section{Situational leadership theory}

\subsection*{Fiedler Contingency Model}
가장 일반적인 상황 이론은 Fred Fiedler에 의해 개발되었습니다. Fiedler는 개인의
리더십 스타일이 수명 전반을 통해 형성되며 따라서 극도로 변경하기 어렵다고
믿었습니다. Fiedler는 사람들이 특정 리더십 스타일을 이해하고 그 스타일을 특정
상황에 맞추는 방법을 가르치는 대신 특정 리더십 스타일을 가르치는 것보다 사람들을
돕는 데 집중해야 한다고 주장했습니다. Fiedler는 특정 리더십 스타일을 이해하는 데
도움이 되도록 최소 선호 동료(LPC) 척도를 개발했습니다. Fiedler에 따르면 리더십
행동이 고정되어 있기 때문에 효과성은 과제를 재구성하거나 리더가 조직적
요인(급여, 징계 조치, 승진 등)에 대한 권한을 변경함으로써만 향상될 수
있다고합니다.

Fiedler의 모델에는 일부 약점이 있습니다. 예를 들어, 어떤 리더는 다른
상황에서보다 더 효과적 일 수 있습니다. LPC 척도는 평가가 한 사람에 의해 다른
사람에게 수행되기 때문에 의문을 제기할 수 있습니다.

이 이론은 작업 그룹이나 조직의 효과성은 주로 두 가지 주요 요인에 달려있다고
주장합니다: 지도자의 성격과 상황이 지도자에게 권력, 통제 및 영향력을 부여하는
정도, 또는 그 반대로 상황이 지도자에게 불확실성을 직면시키는 정도에 달려
있습니다.

Fiedler에게 있어서 스트레스는 지도자의 효과성에 있어서 핵심 결정 요소이며,
지도자의 상관이나 부하자 또는 상황 자체와 관련된 스트레스 사이에 구분이
이루어집니다. 스트레스가 있는 상황에서 지도자는 다른 사람들과의 스트레스 관계에
집착하고 직무에 대한 지적 능력을 집중할 수 없습니다. 따라서 지능은 스트레스가
없는 상황에서 더 효과적이며 더 자주 사용됩니다. Fiedler는 경험이 낮은 스트레스
상태에서의 성과를 저해하지만 고스트레스 상황에서의 성과에 기여한다고
결론지었습니다. 다른 상황 요인들과 마찬가지로, 스트레스가 있는 상황에서
Fiedler는 지도자의 강점을 활용하기 위해 리더십 상황을 변경하거나 조정하는 것을
권장합니다.

Fiedler의 상황적 대조 이론은 그룹의 효과성이 지도자의 스타일(본질적으로 특성
측정)과 상황의 요구 사항 사이의 적절한 일치에 달려있다고 주장합니다. 다시 말해,
효과적인 리더십은 리더의 스타일을 적절한 환경에 맞추는 것에 의존합니다.
Fiedler는 상황적 제어를 지도자가 그룹이 무엇을 할 것인지 결정할 수 있는 정도로
지도자 행동의 효과를 결정하는 주요한 대비 요소로 간주합니다.

Fiedler의 대조 모델은 그룹이 직면한 현재 상황과 지도자의 개인적인 특성과 동기가
상호작용한다고 말하는 동적 모델입니다. 따라서 대조 모델은 리더십 효과를 개인적
성격에만 귀속시키는 경향에서 벗어나는 것을 나타냅니다.

\begin{quote}
\end{quote}

\subsection*{Path-Goal Theory}
리더의 역할은 하급자들이 높은 목표를 세우고 자신감을 가지고 노력하여 성공적으로
과업을 수행함으로써 원하는 보상을 받을 수 있도록 길을 명확히 해주는 것이다. 즉,
이 이론은 리더의 동기부여기능을 강조한다. 길을 안내하는 과정에서 리더가
하급자에게 보여줄 수 있는 리더십 행동은 네 가지가 있다.
\begin{itemize}
  \item 지시적 리더십: 직무를 명확히 해줌. 규정, 일정을 수립해줌.
  \item 후원적 리더십: 욕구와 복지를 우선하고, 친구나 동지처럼 대함.
  \item 참여적 리더십: 의사결정시 의견을 듣고 반영함.
  \item 성취지향적 리더십: 목표에 자신감을 가지고 도전하도록 도움.
\end{itemize}
이 4가지 리더십 행동들이 적합한 상황과 그에 맞는 행동은 다음과 같다.
\begin{itemize}
  \item 추종자의 자신감 결여 $\to$ 후원적 리더십
  \item 직무가 모호한 상태 $\to$ 지시적 리더십
  \item 직무가 도전적이지 않음 $\to$ 성취지향적 리더십
  \item 부적절한 보상 $\to$ 참여적 리더십
\end{itemize}

\subsection*{Performance Readiness Levels}
하나의 상황변수만을 사용하는 간결한 이론으로 하급자의 성숙도 능력의 고저, 의지의
고저에 따라 리더십 스타일을 다르게 해야 한다.
\begin{itemize}
  \item 능력 낮음, 의지 낮음 $\to$ 지시적 리더십: 역할 정해주기
  \item 능력 낮음, 의지 높음 $\to$ 설득형 리더십: 지원을 위한 상담
  \item 능력 높음, 의지 낮음 $\to$ 참여형 리더십: 칭찬과 관심을 기울임
  \item 능력 높음, 의지 높음 $\to$ 위임형 리더십: 관한 위임
\end{itemize}

\subsection*{Vroom-Yetton Decision Model}

참여에는 제도적으로 의사결정에 구성원들의 참여를 보장하는 방법과 리더가 보다
효과적으로 리더십을 발휘하기 위해 자발적으로 하급자들을을 참여시키는 것이 있다.
전자는 제도적 참여, 후자가 참여적 리더십이다. 참여적 리더십의 유형은 다음과 같다.
\begin{itemize}
  \item 순수독단형: 리더가 스스로 의사결정을 내림.
  \item 참고적독단형: 팔로워들로부터 단순 정보를 얻되, 스스로 의사결정을 내림. (의견 받지 않음.)
  \item 개별참여형: 팔로워들로부터 해결책에 대한 의견을 얻되, 그룹으로 접촉하지 않으며, 리더가 의사결정을 내림.
  \item 집단참여형: 팔로워들이 그룹으로 만나 모은 의견을 얻되, 리더가 의사결정을 내림.
  \item 위임형: 팔로워들에게 의사결정권을 주되, 리더가 도와줌.(방임이 아님)
\end{itemize}

적합한 의사결정을 하기 위해, 문제의 속성, 리더 속성, 하급자 속성을 파악해야
한다. 문제가 얼마나 중요한가? 리더가 얼마나 얼만큼의 정보를 가지고 있는가?
팔로워의 의견차이가 얼마나 있는지 등이 있다.

참여의 7가지 법칙은 다음과 같다.
\begin{itemize}
  \item 리더 정보의 법칙: 리더가 정보가 없으면 순수독단형은 사용하지 말라.
  \item 목적 일치의 법칙: 팔로워가 조직의 목표를 중시하지 않으면 위임은 사용하지 말라.
  \item 구조 부조의 법칙: 중요한 문제인데 문제가 구조화되어있지 않으면, 집단참여/위임하라.
  \item 수용의 법칙: 팔로워의 수용이 중요한 문제인데 일방적 결정을 받아들일 것 같지 않으면, 독단하지 말라.
  \item 수용이 중요한데 의견 일치가 안된다면, 집단참여/위임해라.
  \item 공평성의 법칙: 비교적 사소한 문제인데 수용이 중요하면 위임하라.
  \item 수용우선의 법칙: 수용이 중요한데 팔로워가 조직목표가 일치하면 위임하라.
\end{itemize}

심리적 임파워먼트란 개인이 업무의 중요한 측면에 파워를 가진듯이 느끼고 행동하게 되는 과정이다.
심리적 임파워먼트의 4가지 차원에는 의미감, 자기결정성, 자기 효능감, 영향력이 있다.