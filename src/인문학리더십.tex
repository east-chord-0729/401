\chapter{Humanities Leadership}

ttakala@hanmail.net

과제 3개

출석 20점 자기소개 10점 중간고사 30점 - 한글 워드 기준 3장 줄간격 글자크기 160과
10 "자유로운" 에세이. 챗gpt금지. 기말고사 40점 - 한글 워드 기준 4장 제출은 pdf로

10점짜리 자기소개 - 한글 워드 기준 1장 채우기. 글자 10pt 줄간격 160

사진, 전화번호, 수업 신청 배경 (사진 포함 1장임.)

조장이 하는 일. 조원소개서, 토론 10분 - 조장 중심 발표.

조원소개서: 이름, 학과, 수업 신청 배경

\section{2주차}
메데이아 리더십, 스파르타쿠스 리더십

\subsection{메데이아}

[인물관계도]


아이손(이올코스 왕)     아이에테스(콜키스 왕)      크레온(코린토스 왕)
  |                 |                       |
  |                 |                       |
이아손  ----------  메데이아                  크레우사
            |
            |
          아들 2명

그리스인 이아손(이올코스의 왕 아이손의 아들). 콜키스에 금이 많아 식민지를
건설하려 함. -> 원정대 여정

펠리아스가 이아손 죽이려고 콜키스 황금모피를 가져오라함. (왕위를 물려주기
싫었음.)

콜키스 왕 아이에테스에 보물(황금모피)을 달라고 함. -> 미친놈인가? -> 불을 뿜는
황소를 사용해서 밭을 갈라고 함. + 용의 이빨을 뿌리라고 함.(백골의 용사가 나옴
ㄷ)

이 때 메데이아는 이아손에게 연고를 줌(불 방지) + 백골용사를 막는 부적 줌. 용을
제거(혹은 재우고)하고 황금모피 이아손한테 줌. + 결혼 약속.

모피 가져왔지만 왕위 안줌. -> 메데이아 빡침 -> 펠리아스 죽이고 코린토스로
쫒겨남.

코린토스 왕 크레온. -> 메데이아와 이혼하면 왕위를 주겠다 함. -> 메데이아를
칼같이 배신.

결국 이아손과 크레우사가 결혼, 메데이아는 개빡침 

복수 시작. 왕(크레온)과 아내(크레우사, 크레온 딸)를 불 붙는 드레스로 죽임. 
자신의 아들 2명(이아손의
아들이므로)를 칼로 죽임.

메데이아는 이아손을 조롱하며 떠남.

\subsection{스파르타쿠스}
스파르타쿠스: 트라키아 출신.

\section{3주차}

수업목표: 목표지향적 리더십.

코카서스 3국 역사.

해시오도스 해석, 아이스킬로스 해석 2개.

\subsection{코카서스}
조지아. 흑해 옆. 최초 포도주 생산 지역으로 알려져 있음. 카즈베기산
아르마니아인(노아의 후손), 아제르바이잔인(불의 땅 - 땅에서 가스때문에 불이 자연스럽게 나옴.)

프로메테우스는 코카서스의 바위산에서 형벌을 받음.

프로메테우스 : 티탄족과 바다요정의 후손, 티탄 신들의 심부름꾼.
능력 : 미래예지, 제작(장인), 저항(제우스에 저항)

헤시오도스의 해석(일반적인 해석): 제우스는 선, 프로메테우스는 악.

제우스를 기만(제물에 장난을 침), 인간에게 불을 줌.-이게 왜? 

프로메테우스는 인간남자를 만듬. 여성은 제우스가 여자(판도라)를 만듬. - 제우스가 인간에게 내리는 벌.
둘이 결혼해서 자식을 낳음. 그것이 인류가 만들어진 이유.
판도라 속 희망도 사실 악한 것이다. -> 과도한 희망 = 불행

독재에 저항하고, 존엄성을 지키는 프로메테우스와 코버첸코(영화 비스트)의 공통점.
김동현: 자기 목숨보다 소중한 무언가가 있어야 하지 않을까? 자기희생정신이 있어야 한다.
이창연: 본인의 신념. -> 자기 희생. 존엄성이 신념이 깔려있었고, 자기의 신념에 반하는 일.
박채우: 프로메테우스 인간을 행복하게 해주기 위함. 자기행복보다는 남의 존엄성을 지켜줌.
장지연: 사회적 관습이 자신이 생각 사이에서 타협하지않고 저항하는 모습.
이진현: 저항하기 힘든 상황에서 자신만의 뚜렷한 관점과 신념이 있기에 저항할 수 있었다.

아이스킬로스 해석

제우스와 프로메테우스의 갈등 구조만 보자. 이전에는 제우스-선, 프로메테우스-악이었지만,
여기서는 프로메테우스를 선으로 본다. 제우스는 자신의 미래를 알고 싶어 프로메테우스에게 요청. 그러나 거절.
화딱지 나버림. 

프로메테우스는 인간을 도왔음. 그리고 제우스의 비밀(테티스 여신을 인간과
결혼시킴. (신과 결혼하면 자신을 능가하기 때문))을 함구.

25. 고통받는 인간 여성 (고통받는 남성 신과 비슷.) 
프로메테우스가 이오를 보니깐, 나중에 제우스의 부인이 되는 걸 봄.
지금은 고통받고 있지만, 미래엔 잘 될꺼야. 라며 위로해줌.


역사적 인물들의 비전 중 가장 인상 깊었던 인물.
이순신: 높은 관료들이 고문을 줌. 배도 불태워지고, 그 상황속에서 중꺽마. 
이순신은 억까를 당하는 상황속에서도 나라를 지키겠다는 비전을 가짐.
세종대왕: 글자를 알면 똑똑해져서 안된다. 그러나 글자를 만들었죠.
김유신: 삼국통일. 원래 귀족 출신이 아닌데, 삼국통일을 하겠다는 의지하나로 전쟁에 참여해서 삼국을 통일했다.
스티브 잡스: 애플 CEO, 회사가 잘돌아가려면 시스템이 잘 갖추어져야 하는데, 시스템을 구성할 때, 기준이
자기가 예외가 아닌 등의 시스템을 구성함.

%******************************************************************************%
\section{컨버전스 리더십}

\subsection{헤라클레스}

올림푸스 신과 기간테스의 대전쟁에서, 운명의 여신은 인간의 도움을 받으면
기간테스를 이길 것이라 예언한다. 제우스는 인간 영웅을 만들기 위해 알크메네라는
인간 여성과 관계를 맺었고, 그렇게 태어난 아이가 헤라클레스이다. 기간테스와의
전쟁 후, 헤라클레스는 헤베와 결혼한다. 헤라클레스는 알크메네와 함께 테베로
정착한다. 오르코메노스와 테베와의 전쟁 중, 헤라클레스의 큰 활약으로 테베가
승리하게 되고, 테베의 공주 메가라와 결혼한다. 헤라클레스를 싫어하던 헤라는
헤라클레스에게 광기를 일으켜 아내와 아들을 죽이게 만든다. 정신을 되찾은
헤라클레스는 자살하려 했으나, 12가지의 과업을 완수하여 속죄하라는 신탁을 받는다.

이 과업은 인간이 자연을 정복하고 통합해가는 일련의 과정으로 이해 가능하면서,
동시에 인간 내면에 감추어진 야수적 권력욕과 동물적 성욕을 제거하고, 폭력성과
예측불가능성을 잠재우고자 하는 인격 수양의 과정으로도 이해가능하다.

\subsection{로물루스}

로마 건국. 뭐 어쩌라고.

\subsection{헤라클레스의 12과제}

헤라클레스의 첫 번째 과업은 사자를 죽이는 것이다. 사자는 야수적 권력 욕망을
뜻한다. 두 번째 과업은 뱀(히드라)을 죽이는 것이다. 죽여도 계속 재생하자, 불을
붙여 재생하지 못하게 하고, 불사의 머리는 바위로 눌러 움직히지 못하게 했다. 뱀은
성적 욕망을 뜻하며, 없애더라도 다시 생기기에 욕망을 없애는 것이 아닌 억제하는
것으로 해석할 수 있다. 세 번째 과제는 암사슴을 생포하는 것이다. 암사슴은
자연변화, 농사를 뜻한다. 이는 인간이 이용해야 하는 것이기에 죽이는 것이 아닌
생포하는 것으로 과업이 되었다. 네 번째 과업인 돼지(야수성)도 생포하는 것이다.

다섯 번째 과업부터는 조금 달라지는데, 아우게이아스의 외양간을 청소하는 것이다.
아우게이아스는 약 3000마리의 동물이 있으며, 약 30년간 청소하지 않은 외양간이다.
이는 문명의 오염을 청소하는 뜻을 담고 있다. 헤라클레스는 인근 강의 물을 끌어와
마굿간을 청소하고, 감사의 표시로 소 300마리를 선물받았다.

여섯 번째 과업은 식인 괴조, 스팀팔로스 새를 퇴치하는 것이다. 이 새는 인간이 가진
부정적 이상을 뜻한다. 헤라클레스는 징과 노래를 이용하여 식인괴조를 놀래켜 하늘로
올려보내고 활로 쏴 죽인다.

일곱, 여덟, 아홉, 열 번째 과제는 각각 크레타 황소, 디오메데스 야생마, 히폴리테
허리띠, 게리온 황소때를 생포 혹은 훔치는 것이다. 이는 각각 남쪽, 북쪽, 동쪽,
서쪽에 원정을 나가는 것을 의미하며, 그리스의 세력 확장 이야기가 반영된 것이다.

열한 번째 과제와 열두 번째 과제는 원래 열 개의 과제에서 추가된 것이다. 열한 번째
과제는 헤스페리데스의 사과를 가져오는 것이다. 이 사과는 힘과 기지를 뜻한다.
헤스페리데스의 사과는 제우스와 헤라의 결혼 증표로, 이는 용이 지키고 있다. 
아틀라스는 사과를 주는 대신, 자기 대신 하늘을 떠받쳐달라는 거래를 제시헀다.
헤라클래스는 이에 응하는 척하며, 사과만 가져간다.

열두 번째 과제는 하데스의 케르베로스를 생포하는 것으로, 이는 저승세계를 상징한다.
이는 그리스 영웅이 삶과 죽음의 구조에 직접 개입할 수 있다는 것을 보여준다. 
헤라클레스는 완력을 사용하여 케르베로스를 지상으로 끌고온다.

토론:
스티브잡스: PPT -> mp3 전화 인터넷 하나로 융합해서 스마트폰
손흥민: 최근에 이강인과의 불화. 손흥민이 이를 감싸주면서 팀적으로 더 돈독해지는 사건이 있었다.
빌게이츠: 컴퓨터 기술 발전, 교육 개선, 빈곤층 해결 등 전체적으로 모든 것을 해결한 사람.
명량해전: 이순신, 위기상황에서 우리나라 병사들을 잘 이끌어나가 상황을 해결하는 것으로.
12가지의 의미에 대한 의견 --> 과업이라고 하지만, 이는 성장해나가는 과정. 과업을 통해 도전과 참회 속죄를 통해 깨달음을 얻는
헤라클레스의 인생은 우리의 인생과 비슷하다.

\section{Odysseus}

유재석: 20년 동안 술, 담배를 안하고 예능인으로써 살아감.
간디 : 이유를 붙이지 않을께요.
넬슨 만델라: 인권 운동하시는 분.
안중근:    독립운동을 위해 노력하셨음.
독립운동가: 유관순열사, 친일이면 몸이 편했을텐데 참아내고 독립을 위해 노력함.

운동가: 절제 없으면 못함.

오디세우스가 아직 왕권을 얻지 못한 상황.

[오디세우스의 복수극]
13: Alkinoos 왕은 Odysseus에게 풍성한 선물과 함께 고향 Ithake로 보내줌.
    Athena 여신, Odysseus를 늙은 거지의 모습으로 변신.
14: 변장한 Odysseus, 돼지치기 Eumaios 찾아감.
15: Telemachos, 메넬라오스와 헬레네를 떠나 Ithake로 돌아와 돼지치기 Eumaios 찾아감.
16: Odysseus, Telemachos에게 아버지임을 밝힘.
17: Odysseus가 기르던 개 Argos는 주인을 알아보고는 곧 죽음.
    음식을 구걸하나, Antinoos는 음식 대신 상을 집어 던짐. 
18: 거지 Iros와 권투시합. Odysseus, 하녀 Melantho 꾸짖으나(내통자임), 오히려 비웃음(거지주제에).
19: 늙은 유모 Eurykleia는 오뒷세우스의 발을 씻겨주다가 그가 왕임을 앎.
20: 구혼자들과 Odysseus와의 긴장이 고조되고, 복수할 준비 함.
    예언자 Theoklymenos는 구혼자들 모두에게 죽음의 징표가 드리워져 있다고 말함.
21: Penelope, 구혼자들에게 활쏘기 시합 제안, 구혼자들이 실패하나, Odysseus 성공.
22: 오뒤세우스, 텔레마코스, 에우마이오스, 소치기 Philoitios의 도움을 받아, 
    108명의 구혼자들 살해(2명은 용서), 염소치기 Melanthios는 구혼자들 편을 들다가 
    고문 받다 죽고, 12명의 하녀들은 교수형 당함
23: Penelope, Eurykleia에게서 Odysseus의 귀향소식 듣지만 불신. 
    움직일 수 없는 침대를 옮겨 달라는 Penelope의 말에 Odysseus가 화를 내자 그제서야 남편임을 인지.
24: 저승(속편) 이야기. Hermes, 구혼자들의 영혼을 지하세계로 인도.
    죽은 구혼자의 가족들이 Odysseus에게 복수하고자 하나, 실패함.   
    Athena, Odysseus 측과 구혼자의 가족들 측의 화해 중재함.
    모든 신들이 Odysseus 지지 선언하나, Poseidon은 거부, Zeus가 구혼자 가족들에게 화해하기를 종용
    Zeus의 벼락을 무서워한 구혼자 유족들은 
    울며 겨자 먹기로 Odysseus에게 막대한 배상금을 물고 평화조약에 서명

계획 수립과 인내, 그리고 절제.
복수극 = 혼내야 할 사람 앞에서 절제 = 감정 조절.
아무거나.
복수극 = 머리는 차갑게 가슴은 뜨겁게
일상에서 복수할 만한 건 없지. 짜증나는 상황에서 인내하는 것, 남을 이해하는 것.
비슷하게 이성을 잃지 않는게 중요. 감정적으로 하면 안된다. 이성적으로 계획을 세우고 실천하는 것.

%******************************************************************************%
\section{Aias}

아킬레우스처럼 사는 것이 아이아스의 삶의 의미.
그리스 군을 위해 희생.
그러나 아킬레우스의 무구를 오디세우스에게 줌. 삶의 의미 상실.

로고스는 자신이 되고 싶은 것을 추구해 성취시키는 훈련.
싸움에 대한 명예를 항상 가지려고 함. 그러나 시대는 그런 명예를 세워주지 않음.


아이아스와 이준구의 공통점:
1. 시야가 좁다. 큰그림을 못본다: 일 냈다가 망쳤다.
2. 방심했다. 나는 될 줄 알았지만 경쟁자가 치고들어온걸 막지 못했다.
3. 자만했다. 당연히 자기가 될 줄 알았는데 안되니까 더 막지 못하는
4. 자신의 목표를 향해 달려갔다. 무구와 회장자리를 얻기 위해.
5. 중요한 이벤트를 앞두고 리더십이 떨어지는 모습을 보였다. 내가 이 리더를 따라도 되는지. -> 명예가 실추되었다.

오디세우스 카이사르 칸베에 롬멜 이순신 등등등
이사람들의 공통점. -> 지적능력이 뛰어남.
중요한 것은 지적 능력이 뛰어나야 전투승리 조직관리 상황판단: 후퇴, 협상 등등.
로고스 = 지적능력. 

칸베에: 이길 보스를 선택을 잘했다. 위기 극복.
이순신은 전쟁 지형을 잘 파악해서 위기 극복. 롤멜을 박사 아저씨가 잘 설명해줌.
생각을 조금만 더 해보면 된다: 롬멜, 이순신, 칸베에: 빠른 상황판단 -> 빠른작전계획이 바로 세워짐.
롤멜: 작전, 교란, 사전 분석 -> 전쟁을 이김. 많은 정보수집과 작전수립으로 전쟁을 승리함.
칸베에: 2인자보다는 리더가 낫지 않나. 자신의 역량을 발휘할수 있는 리더의 위치에 있는게 낫다.

부하들한테 보상을 많이 해줌. -> 충성심 업


1.  핸드셰이크 과정에서 서버가 클라이언트에게 인증서를 전송하는 과정이 있는데, 
반대로 클라이언트가 서버에게 인증서를 전송하는 단계는 없는가? 

2.  BEAST 공격 기법에서, 'HTTP Request의 쿠키 값을 복구하여 인증 토큰을 알아낼 수 있음'
    에 대해서 자세히 설명해 줄 수 있는가?
    (쿠키란?, 인증 토큰이란? 이것으로 어떻게 클라이언트를 사칭하는가 등의 기본적인 것만)

3.  SSL과 TLS은 서로 다른 통신 프로토콜인가?
    만약 다르다면 이 둘의 큰 차이점이 무엇이 있는가?

4.  공격 기법 POODLE 에서 세션 키는 레코드 프로토콜에 사용되는 암/복호화 키인가?
    아니면 키 교환 과정에서 사용하는 개인키인가?
    그리고 왜 세션 키라고 부르는가?

    