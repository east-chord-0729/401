\chapter{인문학 리더십}

ttakala@hanmail.net

과제 3개

출석 20점 자기소개 10점 중간고사 30점 - 한글 워드 기준 3장 줄간격 글자크기 160과
10 "자유로운" 에세이. 챗gpt금지. 기말고사 40점 - 한글 워드 기준 4장 제출은 pdf로

10점짜리 자기소개 - 한글 워드 기준 1장 채우기. 글자 10pt 줄간격 160

사진, 전화번호, 수업 신청 배경 (사진 포함 1장임.)

조장이 하는 일. 조원소개서, 토론 10분 - 조장 중심 발표.

조원소개서: 이름, 학과, 수업 신청 배경

\section{2주차}
메데이아 리더십, 스파르타쿠스 리더십

\subsection{메데이아}
그리스인 이아손(이올코스의 왕 아이손의 아들). 콜키스에 금이 많아 식민지를
건설하려 함. -> 원정대 여정

펠리아스가 이아손 죽이려고 콜키스 황금모피를 가져오라함. (왕위를 물려주기
싫었음.)

콜키스 왕 아이에테스에 보물(황금모피)을 달라고 함. -> 미친놈인가? -> 불을 뿜는
황소를 사용해서 밭을 갈라고 함. + 용의 이빨을 뿌리라고 함.(백골의 용사가 나옴
ㄷ)

이 때 메데이아는 이아손에게 연고를 줌(불 방지) + 백골용사를 막는 부적 줌. 용을
제거(혹은 재우고)하고 황금모피 이아손한테 줌. + 결혼 약속.

모피 가져왔지만 왕위 안줌. -> 메데이아 빡침 -> 펠리아스 죽이고 코린토스로
쫒겨남.

코린토스 왕 크레온. -> 메데이아와 이혼하면 왕위를 주겠다 함. -> 메데이아를
칼같이 배신.

결국 이아손과 크레우사가 결혼, 메데이아는 개빡침 

복수 시작. 왕(크레온)과 아내(크레우사, 크레온 딸)를 불 붙는 드레스로 죽임. 
자신의 아들 2명(이아손의
아들이므로)를 칼로 죽임.

메데이아는 이아손을 조롱하며 떠남.

\subsection{스파르타쿠스}
스파르타쿠스: 트라키아 출신.

\section{3주차}

수업목표: 목표지향적 리더십.

코카서스 3국 역사.

해시오도스 해석, 아이스킬로스 해석 2개.

\subsection{코카서스}
조지아. 흑해 옆. 최초 포도주 생산 지역으로 알려져 있음. 카즈베기산
아르마니아인(노아의 후손), 아제르바이잔인(불의 땅 - 땅에서 가스때문에 불이 자연스럽게 나옴.)

프로메테우스는 코카서스의 바위산에서 형벌을 받음.

프로메테우스 : 티탄족과 바다요정의 후손, 티탄 신들의 심부름꾼.
능력 : 미래예지, 제작(장인), 저항(제우스에 저항)

헤시오도스의 해석(일반적인 해석): 제우스는 선, 프로메테우스는 악.

제우스를 기만(제물에 장난을 침), 인간에게 불을 줌.-이게 왜? 

프로메테우스는 인간남자를 만듬. 여성은 제우스가 여자(판도라)를 만듬. - 제우스가 인간에게 내리는 벌.
둘이 결혼해서 자식을 낳음. 그것이 인류가 만들어진 이유.
판도라 속 희망도 사실 악한 것이다. -> 과도한 희망 = 불행

독재에 저항하고, 존엄성을 지키는 프로메테우스와 코버첸코(영화 비스트)의 공통점.
김동현: 자기 목숨보다 소중한 무언가가 있어야 하지 않을까? 자기희생정신이 있어야 한다.
이창연: 본인의 신념. -> 자기 희생. 존엄성이 신념이 깔려있었고, 자기의 신념에 반하는 일.
박채우: 프로메테우스 인간을 행복하게 해주기 위함. 자기행복보다는 남의 존엄성을 지켜줌.
장지연: 사회적 관습이 자신이 생각 사이에서 타협하지않고 저항하는 모습.
이진현: 저항하기 힘든 상황에서 자신만의 뚜렷한 관점과 신념이 있기에 저항할 수 있었다.