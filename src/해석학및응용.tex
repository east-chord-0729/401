\chapter{해석학 및 응용}

% 단답형 질문은 쪽지, 복잡한 질문은 e-mail. (평일 9:00 ~ 17:00), 학번 이름 학과 분반 밝힐 것.
% e-mail: kagness@kookmin.ac.kr
% 40 40 과제 10 출석(호명) 10
% 쉬는날 비대면.
% 과제는 자필로 + 종이로. 중간 5번 기말 5번 있음. 풀이과정, 노력점수 있음.
% 지각 -1점
% 결석: 주임교수, 대학장 직인필.

\section{Axiom of Real Number}
실수 $\RR$는 체의 공리 \ref{axm:field}, 순서의 공리, 완비성의 공리 세 가지를 따른다.

\subsection*{Axiom of Field}

실수의 체 공리를 설명 하기 전에, 체(field)에 대해 먼저 정의한다.

\begin{definition}
  다음이 성립하는 집합 $\FF$를 체라고 한다.
  \begin{itemize}
    \item 덧셈: 교환 법칙, 결합 법칙, $0$이 존재, 역원이 존재.
    \item 곱셈: 교환 법칙, 결합 법칙, $1$이 존재, 역원이 존재.
    \item 분배 법칙.
  \end{itemize}
  \label{def:field}
\end{definition}

다음은 실수의 체 공리이다.

\begin{axiom}
  $\RR$은 체이다.
  \label{axm:field}
\end{axiom}

\subsection*{Axiom of Order}

% http://mathonline.wikidot.com/theorems-on-the-order-properties-of-the-real-numbers

\begin{axiom}
  Let $P$ be a nonempty subset of $\RR$ which we define as the set of positive
  real numbers. Then $P$ have following axioms:
  \begin{itemize}
    \item Axiom 1. If $a, b \in P$, then $a + b \in P$.
    \item Axiom 2. If $a, b \in P$, then $a \cdot b \in P$.
    \item Axiom 3. If $a \in \RR$ then only one of the following holds: $a \in P$
          or $-a \in P$ or $a = 0$.
  \end{itemize}
\end{axiom}

\begin{property}
  Let $P$ be a nonempty subset of $\RR$ which we define as the set of positive
  real numbers. Then $P$ have following properties:
  \begin{itemize}
    \item $\forall a, b, c \in \RR$, exactly one of $a < b$, $a = b$ and $a > b$
          is true. 순서가 있으니 $<, >, =, \ge, \le$ 등을 사용할 수 있음.
    \item $\forall a, b, c \in \RR$, if $a < b$ and $b < c$ then $a < c$.
    \item $\forall a, b \in \RR$, if $a \le b$ and $a \ge b$ then $a = b$.
    \item $\forall a, b, c \in \RR$, if $a < b$ then $a + c < b + c$.
    \item $\forall a, b, c \in \RR$, if $a < b$ and $c > 0$ then $ac < bc$.
  \end{itemize}
\end{property}

\subsection*{Axiom of Completeness}
이 장에서 항상 $X$는 $\RR$의 공집합이 아닌 부분 집합이라 하자. 완비성의 공리를 설명하기 전에,
유계, 상계 그리고 상한에 대해 정의한다.

\begin{definition}
  모든 $x \in X$에 대하여, $a \geq x$를 만족하는 $a \in \RR$가 존재 할 때, $X$를
  위로 유계(bounded above)라 하고, $a$를 $X$의 상계(upper bound)이라 한다.
  \label{def:bound}
\end{definition}

\begin{definition}
  $a$가 $X$의 상계이고 모든 $X$의 상계 $a'$에 대해 $a \leq a'$라면,
  $a$를 $X$의 상한(supremum or least upper bound)이라 한다.
  기호로는 $\sup X$로 나타낸다.
  \label{def:supremum}
\end{definition}

% $X = (-\infty, 10)$일 때, 최댓값은 없고, $\sup X = 10$이다.

완비성의 공리는 다음과 같다.

\begin{axiom}[Completeness]
  $X$가 위로 유계일 때, $\sup X$는 존재하며, 유일하다. (유일하다는 공리가 아닐수도.)
  \label{pro:bound2}
\end{axiom}

\subsection*{?}

\begin{proposition}
  $X$가 위로 유계이고, $a$가 $X$의 상계일 때, $a = \sup X$라면, 모든 $\eps < 0$에 대해
  $a - \eps < x \leq a$를 만족하는 $x \in X$가 존재하며, 그 역도 성립한다.
  \label{pro:bound1}
\end{proposition}

\begin{proof}
  $(\rif)$ 여기서는 대우 명제를 증명한다. 모든 $\eps < 0$에 대해 $a -
    \eps < x \leq a$를 만족하는 $x$가 존재하지 않는다고 가정하자. 이는 즉
  모든 $x$에 대해 $a - \eps < x$를 만족하므로, $a - \eps$은 $X$의
  상계가 된다. $a - \eps < a$이기 때문에 $a$는 $\sup X$가 아니다. \\
  $(\lif)$ 모든 $\eps < 0$에 대해 $a - \eps < x \leq a$인 $x$가
  존재하므로 $a - \eps$은 상계가 아니다. 이는 $a$보다 작은 모든 실수가
  상계가 아니라는 뜻이므로, $a$는 상한이다.
\end{proof}

$\sup X = \infty$라면, $X$는 위로 유계가 아니다.

$X = \emptyset$이면, $\sup = -\infty$.
$\forall r \in \RR$, $x \in X \implies x \leq r$이다.(p이면 q이다 에서 p가
거짓이면, 이 조건문은 무조건 참).

\begin{definition}
  $X$가 위로 유계가 아니면, $\sup X = \infty$이라 한다.
  $X$가 아래로 유계가 아니면 $\inf X = -\infty$이라 한다.
  \label{def:not_bound}
\end{definition}

실수 $\RR$은 위로 유계도, 아래로 유계도 아니다.

아르키메데스의 성질
\begin{theorem}
  모든 $a, b \in \RR, a > 0$에 대하여, $na > b$이 성립 되는 적당한 $n \in \NN$가 존재한다.
  \label{thm:arc}
\end{theorem}

\begin{proof}
  (역이 모순이라는 것을 증명)
  모든 자연수 $n$가 $na \le b$를 만족하는 $a > 0$이 존재한다고 가정하자.
  $b$는 $A = \set{na : n \in \NN}$의 상계이고, $A$는 위로 유계이다.
  완비성의 공리에 의해, $\sup A$는 존재한다.
  즉 $(n + 1) \cdot a \le \sup A$를 만족한다. 이는
  $na < \sup A - a$이므로,  $\sup A - a$는 $A$의 상계이다.
  상한의 정의에 의해, $a \le 0$이므로,

  정리 \ref{pro:bound1}에 의해, 모든 $\eps > 0$에 대해
  $\sup A - \eps < n \le \sup A$를 만족하는 $n$이 존재한다.
\end{proof}

이를 통해 알 수 있는 것. $0 < a < b$에 대해 $na > b$이 성립하는 $n$이 존재한다.
$a$가 아무리 작고 $b$가 아무리 커도, $a$를 계속 더하면 $b$를 넘을 수 있음.

자연수 집합 $\NN$은 실수 $\RR$에서 위로 유계가 아니다.


\subsection*{Exercises}

\begin{exercise}
  모든 $n \ge 4$에 대해, $n! > 2^n$임을 보여라.
\end{exercise}

\begin{solution}
  수학적 귀납법으로 보인다.
  $4! = 24 > 2^4$이므로 $n = 4$일 때 성립.
  $k > 4$에 대하여, $k! > 2^k$가 성립한다고 가정하자. 그러면
  $(k + 1)! > 2^k \cdot (k + 1) > 2^k \cdot 2 > 2^(k + 1)$이므로 증명은 완성된다.
\end{solution}

\begin{exercise}

\end{exercise}

\begin{solution}
  $A_2 = A_2^o and A_2^e$이다.
  $\sup A_2^o = \frac{1}{2}$
  $\inf A_2^e = -\frac{1}{2}$ 뭐 대충 이느낌.
\end{solution}

\begin{exercise}
  $A = (-\infty, 3) and (4, 5]$에서의 상한과 하한을 구해라.
\end{exercise}

\begin{solution}
  $\sup A = 5, \inf A  = -\infty$
\end{solution}

\begin{exercise}
  $a, b \in \RR$이라 하자. 임의의 $\eps$에 대하여, $\abs{a-b} < \eps$이면, $a = b$임을 보여라.
\end{exercise}

\begin{solution}
  $A = \set{\eps \in \RR : \eps > 0}$이라 하자. $A = (0, \infty)$이다.
  $\abs{a - b}$는 $A$의 하계이다. $A$는 아래로 유계이다.
  완비성의 공리에 의해, $\inf A$가 유일하게 존재한다. $\inf A \ge \abs{a - b}$이다.
  $\inf A = 0$이다. $\abs{a - b} \ge 0$이므로, $\abs{a - b} = 0$이다.
\end{solution}

\begin{exercise}
  (베르누이 부등식)$x > -1$이면 모든 자연수 $n$에 대하여 $(1 + x)^n \ge 1 + nx$임을 보여라.
\end{exercise}

\begin{solution}
  PMI(?) $(1 + x)^1 = 1 + x \ge 1 + 1x$이므로 성립.
  $(1 + x)^k \ge 1 + kx$가 성립한다고 가정하자.
  $(1 + x)^(k+1) = (1 + x)^k \cdot (1 + x) \ge (1 + kx) \cdot (1 + x)
    = 1 + kx + x + kx^2 \ge 1 + kx + x = 1 + (k + 1)x$가 성립하므로 증명은 완성된다.
\end{solution}

%******************************************************************************%
\newpage
\section{Sequence}

\begin{definition}
  \label{def:limit}
  We call $l$ the \emph{limit} of the sequence $\set{x_n}$, which is written
  $x_n \to l$, if the following condition holds: For each real number $\eps > 0$,
  there exists a natural number $N$ such that, for every natural number $n \ge
    N$, we have $\abs{x_n - l} < \eps$. The sequence $\set{x_n}$ is said to
  \emph{converge} to the limit $l$.
\end{definition}

% 예를 들어, $x_n = \frac{1}{n}$이라고 할 때, 아르키메데스의 성질에 의해 $x_n \to 0$이다.
% $x_n = 2 + \frac{1}{2^n}$이라고 할 때, 베르누이 부등식을 사용하여, $x_n \to 2$임을 알 수 있다.

\begin{theorem}
  \label{thm:limit_unique}
  If the limit of a sequence exists then it is unique.
\end{theorem}

\begin{proof}
  Let $l_1, l_2$ are limit of the sequence $\set{x_n}$. Then we have
  \begin{equation}
    \begin{split}
      &\forall \eps > 0, \exists N_1:
      \forall n \ge N_1, \abs{x_n - l_1} < \frac{\eps}{2} \\
      &\forall \eps > 0, \exists N_2:
      \forall n \ge N_2, \abs{x_n - l_1} < \frac{\eps}{2}.
    \end{split}
  \end{equation}

  Let $N = \max(N_1, N_2)$. Then we have
  \begin{equation}
    \begin{split}
      \forall n \ge N, \abs{x_n - l_1} < \frac{\eps}{2} \\
      \forall n \ge N, \abs{x_n - l_2} < \frac{\eps}{2}.
    \end{split}
  \end{equation}

  Now we completes the proof by following.
  \begin{equation}
    \abs{l_1 - l_2}
    = \abs{l_1 - x_n + x_n - l_2}
    \le \abs{x_n - l_1} + \abs{x_n - l_2}
    < \frac{\eps}{2} + \frac{\eps}{2} = \eps
  \end{equation}
\end{proof}

% \footnote{역은 성립하지 않는다. $\set{(-1)^n}$가 그 예시이다.}

\begin{theorem}
  \label{thm:limit_bound}
  If the limit of a sequence exists then the sequence is bounded.
\end{theorem}

\begin{proof}
  Let $l_1$ is limit of the sequence $\set{x_n}$. Then we have
  \begin{equation}
    \forall \eps > 0, \exists N: \forall n \ge N, \abs{x_n - l} < \eps.
  \end{equation}

  It follows that
  \begin{equation}
    \abs{x_n} = \abs{x_n - l + l} \le \abs{x_n - l} + \abs{l} < \eps + \abs{l}.
  \end{equation}

  Let $M = \max (x_1, \cdots, x_{N-1})$. Then we completes the proof by
  following.
  \begin{equation}
    \forall n, x_n \le \max (M, \eps + \abs{l}).
  \end{equation}
\end{proof}

% \begin{proposition}
%   $x_n$이 유계수열이고 $y_n$이 0으로 수렴하면, 수열 $x_n y_n$은 0으로 수렴한다.
% \end{proposition}

% \begin{proof}
%   모든 $n \in NN$에 대해 $\abs{x_n} \le M$인 $M$이 존재한다. 모든 $\eps > 0, n
%     \in \NN$에 대해 $n \ge N$을 만족하는 $N$이 존재하면, $\abs{y_n - 0} <
%     \frac{\eps}{M}$이다.

%   다음으로 증명은 완성된다.
%   $\abs{x_ny_n - 0} = \abs{x_ny_n} < M \cdot \frac{\eps}{M} < \eps$.
% \end{proof}

% 과제 1번.

% \begin{proof}
%   for all $\eps > 0$, and for all $n \in \NN$, exist $N$ s.t. $n \ge N$ and
%   $\abs{\frac{1}{n^2 + 1} - 0} < \eps$.

%   $n^2 > n^2 > N^2 \implies \frac{1}{n^2 + 1} < \frac{1}{N^2} = \eps$
% \end{proof}

\begin{theorem}[Squeeze theorem]
  \label{thm:limit_squeeze}
  Let $x_n \to l, y_n \to l$. Then
  \begin{equation}
    \forall n, x_n \le z_n \le y_n \implies z_n \to l.
  \end{equation}
\end{theorem}

\begin{proof}
  Since $x_n \to l, y_n \to l$, we have
  \begin{equation}
    \begin{split}
      \forall \eps > 0, \exists N_x: \forall n \ge N_x, -\eps < x_n - l < \eps \\
      \forall \eps > 0, \exists N_y: \forall n \ge N_y, -\eps < y_n - l < \eps.
    \end{split}
  \end{equation}

  Let $N = \max(N_x, N_y)$. since $x_n \le z_n \le y_n$, we completes the proof
  by following
  \begin{equation}
    \forall \eps > 0, n \ge N, -\eps < x_n - l \le z_n - l \le y_n - l < \eps.
  \end{equation}
\end{proof}

% $z_n = \sin / n \to 0$.

% 증가 수열, 감소 수열, 순증가 수열, 순감소 수열, 단조수열.

\subsection{Monotone Sequences}

\begin{definition}
  \label{def:mono}
  the squence $\set{x_n}$ is \emph{monotone increasing sequence} if $\forall n,
    x_n \le x_{n+1}$.
\end{definition}

\begin{lemma}
  If a sequence is increasing and bounded above, then its supremum is the limit.
  \label{lem:mono_limit_1}
\end{lemma}

\begin{proof}
  Let $x := \sup{\set{x_n}}$.
\end{proof}

\begin{lemma}
  If a sequence is decreasing and bounded below, then its infimum is the limit.
  \label{lem:mono_limit_2}
\end{lemma}

\begin{proof}

\end{proof}

\begin{theorem}[Monotone convergence theorem, MCT]
  If a sequence is bounded and monotone, then it converges.
  \label{thm:mono_limit}
\end{theorem}

\begin{proof}

\end{proof}

%***%
\subsection{Subsequences}

\begin{definition}
  $\set{x_{n_k}}$ is a \emph{subsequence} of $\set{x_n}$ if $\set{n_k}$
  is increasing sequence.
  \label{def:sub}
\end{definition}

\begin{theorem}
  If a sequence $\set{x_n}$ has a limit $l$ then all subsequence
  $\set{x_{n_k}}$ of $\set{x_n}$ also have a limit $l$.
  \label{thm:sub_limit}
\end{theorem}

\begin{proof}
  It is clear that $n_1 \ge 1$. Suppose that $n_k
    \ge k$. Since $n_k$ is increasing sequence, we have $n_{k+1} > n_k \ge k$. It
  follows that $n_{k+1} \ge k + 1$ and we have
  \begin{equation}
    n_k \ge k.
  \end{equation}

  Let $x_n \to l$. We completes the proof by following.
  \begin{equation}
    \forall \eps > 0, \exists k: \forall n_k \ge k, \abs{x_{n_k} - l} < \eps.
  \end{equation}
\end{proof}


\subsection{Nested Sequences}

\begin{definition}
  Let $I_n = [a_n, b_n]$, where $\abs{I_n} = b_n - a_n$ denotes the
  \emph{length} of such an interval. One can call $I_n$ a sequence of nested
  intervals, if
  \begin{equation}
    \begin{split}
      \forall n, I_{n + 1} < I_n,
    \end{split}
  \end{equation}
  \label{def:nest}
\end{definition}

\begin{theorem} [Nested intervals theorem]
  Let $I_n = [a_n, b_n]$ is nested. If $(b_n - a_n) \to 0$, then $\exists! x \in
    \RR$ such that $x \in I_n$.
  \label{thm:nest_int}
\end{theorem}

\begin{proof}
  First we prove existence. Note that $a_n$ is bounded. By MCT, $a_n \to \sup
    A$. Let $x = \sup A$. Since $b_n$ is upper bound, $a_n \le x \le b_n$. then $x
    \in I_n$. Now we prove uniqueness. Assume that $y \in I_n$. then $a_n \le y
    \le b_n$. It follows that $0 \le y - a_n \le b_n - a_n$. By squeeze theorem,
  $y - a_n \to 0$. It follows that $a_n \to y = x$.
\end{proof}

\begin{theorem} [Bolzano-Weierstrass theorem]
  bounded sequence has a convergent subsequence.
  \label{thm:nest_bol}
\end{theorem}

\begin{proof}
  Let $\set{x_n}$ is a bounded sequence. Then we have
  \begin{equation}
    \forall n, \exists M > 0: -M \le x_n \le M.
  \end{equation}

  Let $I_1 = [0, M], I_n = [\frac{(2^n - 1)M}{2^n}, M]$. Since $\forall k, I_k
    \supset I_{k + 1}$, $I_n$ is nested. By nested intervals theorem,
  $\abs{x_{n_k} - x} < \frac{M}{2^{n-1}}$.
\end{proof}


\subsection{Cauchy Sequences}

\begin{definition}
  a sequence $\set{x_n}$ is \emph{Cauchy sequence} if
  \begin{equation}
    \forall \eps > 0, \exists N: \forall m > n \ge N, \abs{x_m - x_n} < \eps.
  \end{equation}
  \label{def:cauchy}
\end{definition}

\begin{theorem}
  if a sequence converges, then it is cauchy sequence.
  \label{thm:cauchy_conv}
\end{theorem}

\begin{proof}
  Let $x_n \to l$. Then
  \begin{equation}
    \begin{split}
      \forall \eps > 0, \exists N: \forall n \ge N, 
      \abs{x_n - l} < \frac{\eps}{2}, \\
      \forall \eps > 0, \exists N: \forall m \ge N, 
      \abs{x_n - l} < \frac{\eps}{2}.
    \end{split}
  \end{equation}

  Let $m > n$. Then
  \begin{equation}
    \abs{x_m - x_n}
    =   \abs{x_m - l + l - x_n}
    \ge \abs{x_m - l} + \abs{x_n - l}
    <   \frac{\eps}{2} + \frac{\eps}{2}.
  \end{equation}
\end{proof}

\begin{theorem}
  A cauthy sequence is bounded.
  \label{thm:cauchy_bound}
\end{theorem}

\begin{proof}
  Let $x_n$ is a cauchy sequence. Then
  \begin{equation}
    \exists N: \forall m > n \ge N, \abs{x_m - x_n} < 1.
  \end{equation}

  Let $n < N$. Then
  \begin{equation}
    \abs{x_n} < \max (\abs{x_1}, \cdots, \abs{x_{N-1}}).
  \end{equation}

  Let $n \ge N$. Then
  \begin{equation}
    \abs{x_n} 
    =   \abs{x_n - x_N + x_N} 
    \le \abs{x_n - x_N} + \abs{x_N}
    <   1 + \abs{x_N}.
  \end{equation}

  Let $M = \max (\abs{x_1}, \cdots, \abs{x_{N-1}, 1 - \abs{x_N}})$. Then
  $\exists M: \abs{x_n} \ge M$. This completes the proof.
\end{proof}

\begin{theorem} [Cauchy's convergence test]
  A sequence converges if and only if it is cauchy sequence.
  \label{thm:cauchy_test}
\end{theorem}

\begin{proof}
  % (->) 했음.
  % (<-) 코시수열 은 유계수열. 유계수열이면 수렴하는 부분수열을 가짐(PWT). 

  We claim that $x_{n_k} \to l \implies x_n \to l$. 
  
  Let $\set{x_n}$ is a cauchy sequence and $\set{x_{n_k}}$ is a subsequence of
  $\set{x_n}$. Then
  \begin{equation}
    \begin{split}
      &\forall \eps > 0, \exists N_1: \forall m > n \ge N_1, \abs{x_m - x_n} 
      < \frac{\eps}{2} \\
      &\forall \eps > 0, \exists N_2: \forall n_k > k \ge N_2, \abs{x_{n_k} - l} 
      < \frac{\eps}{2}.
    \end{split}
  \end{equation}

  Let $N = \max (N_1, N_2)$. Then we completes the proof by following:
  \begin{equation}
    \abs{x_n - l} 
    =   \abs{x_n - x_{n_k} + x_{n_k} - l}
    \le \abs{x_n - x_{n_k}} + \abs{x_{n_k} - l}
    <   \frac{\eps}{2} + \frac{\eps}{2}
    =   \eps.
  \end{equation}
\end{proof}

\subsection{Abstract}

\begin{figure}[H]
  \centering
  \begin{tikzpicture}

  \end{tikzpicture}
  \caption{Relations of theorem about sequence}
  \label{fig:}
\end{figure}


\subsection{Exercise}

$x_1 = 2$, $x_{n+1} = (2x_n + 3) / (x_n + 2)$.
\begin{proof}
  Claim: $x_{n+1} - (2x_n + 3) / (x_n + 2) > 0$.

  $x_1 = 2$, $x_{n+1} = (2x_n + 3) / (x_n + 2)  = x_n^2 - 3 / x_n + 2$.

  Claim: $x_n^2 > 3$.

  using induction.

  $x_1^2 = 4 > 3$.

  Assume that $x_k^2 > 3$.

  $x_{k+1}^2 - 3 = (2x_k + 3 / x_k + 2)^2 - 3 = x_k^2 - 3 / (x_k + 2)^2 > 0$.

  $x_n > x_{n+1}$ and $x_n > \sqrt{3}$.

  By MCT, $x_n$ has a limit. Let $x_n \to x$.

  $x_{n+1} \to 2x + 3 / x + 2 \implies x = \sqrt{3}$.
\end{proof}