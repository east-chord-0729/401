\chapter{해석학 및 응용}

% 단답형 질문은 쪽지, 복잡한 질문은 e-mail. (평일 9:00 ~ 17:00), 학번 이름 학과 분반 밝힐 것.
% e-mail: kagness@kookmin.ac.kr
% 40 40 과제 10 출석(호명) 10

\section{1주차}

\subsection{체의 공리}

다음이 성립하는 집합 $\FF$를 체라고 한다.
\begin{itemize}
  \item 덧셈: 교환 법칙, 결합 법칙, $0$이 존재, 역원이 존재.
  \item 곱셈: 교환 법칙, 결합 법칙, $1$이 존재, 역원이 존재.
  \item 분배 법칙.
\end{itemize}
실수 집합 $\RR$은 체이다.

$\RR$에는 다음 두 조건을 만족하는 $P(\neq 0)$가 존재한다.
\subsection{순서의 공리}
\begin{itemize}
  \item 덧셈과 곱셈은 닫힘.
  \item $a \in \RR$에 대해 다음 셋 중 단 하나만 성립. $a \in P, a = 0, -a \in P$.
\end{itemize}
순서가 있으니 $<, >, =$ 등을 사용할 수 있음.

\begin{theorem}
  \label{thm:acbc}
  $\forall a, b, c \in \RR, a > b, c > 0 \implies ac > bc$.
\end{theorem}

\begin{proof}
    $a - b > 0, c > 0$이므로, $ac - bc = (a-b)c > 0$. 따라서 $ac > bc$이다.
\end{proof}

\subsection{시작}
$X$를 $\RR$의 공집합이 아닌 부분 집합이라 하자. 

\begin{definition}
  \label{def:bound}
  $\forall x \in X, a \geq x$인 $a$가 존재 할 때, $X$를 
  위로 유계(bounded above)라 하고, $a$를 $X$의 상계(upper bound)이라 한다.
  $\forall x \in X, b \leq x$인 $b$가 존재 할 때, $X$를 
  아래로 유계(bounded below)라 하고, $b$를 $X$의 하계(lower bound)이라 한다.
\end{definition}

\begin{definition}
  \label{def:supremum}
  $a$가 $X$의 상계이고, $b$가 $X$의 상계일 때, $a \neq b$라면, $a$를 $X$의 상한(supremum)
  또는 (least upper bound)라 한다. 기호로는 $\sup X = a$로 나타낸다.
\end{definition}
$X = (-\infty, 10)$일 때, 최댓값은 없고, $\sup X = 10$이다.