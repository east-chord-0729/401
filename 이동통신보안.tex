\section{이동통신보안}

\subsection*{1주차}

암호화 기본 기법 네 가지. 이걸로 보안 시스템 구현 가능.
\begin{itemize}
  \item 대칭키: $\alice: \ct = \enc(\key, \pt)$, $\alice \to \bob: \id(\alice), 
        \ct$, $\bob: \pt = \dec(\key, \ct)$. $\id$는 그 사람을 식별하기 위함. 
        공개키보다 빠르다. 블록 or 스트림 암호.
        블록의 경우 운영모드가 있음. 전사공격 유일하게 허용. $O(2^{\klen})$. 
        AES, SEED, ARIA, LEA. 키 분배 문제. 확장성 문제($x$명의 경우, $\binom{x}{2}$).
        키 분배 문제: 키분배센터(KDC)라는 제 삼자 이용. KDC가 $\alice$와 $\bob$에게 $\key$
        전달. 커버러스 통신이 이거 사용.
        \begin{equation}
          \begin{split}
            \alice \to \carol&: \id(\alice), \id(\bob) \\
            \carol \to \alice&: \ct_1 \samp \enc_{\key_{ac}}(\key_{ab}), 
            \ct_2 \samp \enc_{\key_{bc}}(\key_{ab}) \\
            \alice&: \key_{ab} \samp \dec_{\key_{ac}}(\ct_1), 
            \ct \samp \enc_{\key_{ab}}(\pt) \\
            \alice \to \bob&: \ct, \ct_2 \\
            \bob&: \key_{ab} \samp \dec_{\key_{bc}}(\ct_2), 
            \pt = \dec_{\key_{ab}}(\ct).
          \end{split}
        \end{equation}
        이 과정의 문제점은, 키가 추가로 필요하다는 점. 추가로 필요한 키도 보호가 필요하다는 점.
        이 문제를 공개키 암호가 해결할 수 있음.
  \item 공개키: 키가 두 개. 비밀키, 공개키. 둘은 수학적 연계.
        $\bob \to \alice: \ct = \enc_{\pkey_{\alice}}(\pt), 
        \alice: \pt = \dec_{\skey_{\alice}}(\ct)$.
        내가 보냈음을 인증 가능. \memo{인증, 부인방지. 공개키를 조작할 수 있나?} 
        $\bob \to \alice: \ct = \enc_{\skey_{\alice}}(\pt), 
        \alice: \pt = \dec_{\pkey_{\alice}}(\ct)$.
        이 둘을 합치면? 
        % P -> L: C, S s.t. C = e(pk(L), Rec), S = e(sk(P), C)
        % L: check C = d(pk(P), S) rec = d(sk(L), C)
        % 실제로는 이렇게 하지는 않고, 전자서명을 사용한다. ... 왜?
  \item 전자서명
  \item 해시
\end{itemize}


\footnote{우리나라는 it 소비 강국. 해커의 공격이 많음.}